% Chapter Template

\chapter{Optimization Results} % Main chapter title

\label{Optimization Results} % Change X to a consecutive number; for referencing this chapter elsewhere, use \ref{ChapterX}

\lhead{Chapter 4. \emph{Optimization Results}} % Change X to a consecutive number; this is for the header on each page - perhaps a shortened title

%----------------------------------------------------------------------------------------
%	SECTION 1
%----------------------------------------------------------------------------------------

This chapter is the results of the CUDA code implementation launched on several different GPUs nodes. The test are performed on various GPUs architectures, which, has different hardware characteristics. Each GPU node is analyzed using the NVIDIA's Visual Profiler, in addition the CUDA kernels are evaluated in performance; throughput, bandwidth, executing and parallel time. Furthermore the results, are analyzed and optimized using the schemes from chapter 3. Lastly the code is executed on the supercomputer ``Piritakua'' of the  University of Guanajuato, Campus Yuriria.

\section{Supercomputer ``Piritakua''}

The experiments are carried out using the supercomputer “Piritakua”. The massive GPU cluster was design and built by Dr. Claudio from the University of Guanajuato Campus Yuriria. The GPU cluster is located at a small town of Mexico, Yuriria. The supercomputer at the Front-end has a 8 core Intel Xeon at 2.4 Ghz, at the back-end several GPU are connected, one NVIDIA Tesla K20, two Tesla M2070 and a GTX 580.

The specifications of the front-end cluster.

\begin{tabular}{ | p{7.1cm}  | l | l | l |}
  \hline
  Processor & Number & Cores & RAM  \\
  \hline
  Servidor Dell Intel Xeon E5620 2.4 GHz & 1 & 8 & 12 GB \\
  \hline
  Servidores HP Proliant SL 350s Gen3 Intel Xeon X5650 2.67 GHz & 2 & 24 & 32 GB \\
  \hline
   Servidores HP Proliant SL 250s Gen8 Intel Xeon E5-2670 2.60 GHz & 3 & 48 &104 GB \\
   \hline
   CPU Xeon Phi  5110p & 1 & 8 & 8 GB\\
   \hline
   CPU Xeon Phi 7120p  & 1 & 8 & 16 GB\\
   \hline
  \end{tabular}


 The CUDA Code was launched on only two CPUs, a high-end laptop with a eight core intel i7-3630QM and a CPU Xeon Phi 7120p from the cluster.
 
 Talk about the Xeon Phi.
 
 When accessing ``Piritakua'' remotely is possible to use all the GPUs available on the cluster.
 The specifications of the GPU connected to the front-end are as follow.

  \begin{tabular}{ |  l  |  l  |  l  |  l  |  l  | l | }
    \hline
    Model & Cores & RAM & DP & SP & Bandwidth \\
    \hline
    Tesla K20m & 2496 & 5GB & 1.17 Tflops & 3.52 Tflops & 208 GB/s \\
   \hline
    Tesla M2070 & 448 & 6GB & 515 Gflops & 1030 Gflops & 150 GB/s \\
   \hline
     Tesla C2050 & 448 & 2.5GB & 512 Gflops & 1030 Gflops & 144 GB/s \\
   \hline
      GeForce GTX 580 & 512 & 1.5GB & 520 Gflops & 1,154 Gflops & 192.2 GB/s \\
   \hline
   GeForce GTX 670MX & 960 & 3GB & 520 Gflops & 1,154 Gflops & 67.2 GB/s \\
   \hline
  \end{tabular}

   The code was launched on all Piritakua's GPUs and on external GeForce GTX 670m, located on a laptop. The "m" stands for the mobil graphic card version. In addition the 670m card is design for less power usage, but with high graphics power, it even has more cores than some Tesla models, however this types of cards has way more less Bandwidth than standard versions.

There are two GPU architectures that code was launched, the Fermi and the Kepler. The Tesla K20m is base on ``Kepler'' GPU architecture and Tesla M2070, Tesla M2050 and GeForce GTX 580 on the Fermi architecture. The Kepler is a newer architecture than the Fermi. The big difference between the is the number of CUDA cores per SM.


 \subsection{Experiment detail}

\section{Results}

The CUDA code was launched on each GPU of the Piritakua supercomputer. As we know the supercomputer has different GPU, as well as several architectures and different number of CUDA cores.



\subsection{Initial Test}



\subsubsection{Visual profiler}

The visual profiler.

The visual profiler was used on Laptop with GeForce GTX 670m with the intel eight core  i7-3630QM.


the


\subsection{Obtimized}


%-----------------------------------
%	SUBSECTION 2
%-----------------------------------

\subsection{Subsection 2}

%----------------------------------------------------------------------------------------
%	SECTION 2
%----------------------------------------------------------------------------------------

\section{Main Section 2}
