% Chapter Template

\chapter{Introduction to Domain Wall Dynamics under Nonlocal STT} % Main chapter title

\label{Introduction to Domain Wall Dynamics under Nonlocal STT} % Change X to a consecutive number; for referencing this chapter elsewhere, use \ref{ChapterX}

\lhead{Chapter 3. \emph{Introduction to Domain Wall Dynamics under Nonlocal STT}}% Change X to a consecutive number; this is for the header on each page - perhaps a shortened title


This chapter is a brief overview of the theory of spintronics and the study of Domain Wall Dynamics under Nonlocal Spin-Transfer Torque. This is a quantitatively test the effects of spin-diffusion, on real Domain Wall (DW) structures, by numerically implementing the Zhang - Li model into a NiFe soft nanostrip. The numerical method used for the solution is a the method known as Finite Differences in the Time Domain (FDTD) on a 3D cell grid with whose integration is done using a 4th order Runge-Kutta integration (RK4).

\section{Theory}

Now as we know, each electron not only carries an elementary
unit of charge $e$, but also carries an elementary unit of angular momentum. Whenever we produce an electrical current by inducing
motions of electrons, it could indeed be viewed as a collection of little magnets that are moving around (see Figure \ref{fig:electron}. In other words, any electron charge transport is simultaneously accompanied by a transport of spin, or magnetic moment carried (passing electron) by these electrons \cite{cornell}.


\subsection{Spintronics}

Spintronics is a new type of electronics that exploits the spin degree of freedom of an electron in addition to its charge \cite{spinz} (see figure \ref{fig:electron}. The interest is motivated by the quest to understand basic physical principles underlying the electron and nuclear spin interactions in materials and by possible technological applications. The field of spintronics has attracted massive interest since the discovery of giant magnectroresistence (GMR) effect in 1988 by Albert Fert and Peter Gr\"{u}nberg who were awarded the 2007 Nobel Prize in physics. The GMR effect has been widely used in hard disk drives (HDD) which have brought a huge impact on industries and consumer electronics. Spintronics is a promising technology which will complement the present electronics with addition "spin" quantum freedom to charge freedom that is currently used in devices \cite{nonlocalspin}.


\subsection{Spin Transfer Torque}

A torque is simply a time rate of change of angular momentum \cite{spintransfer}. Hence, Spin transfer torque or SST occurs when spins flowing from one layer to another can reorient the magnetization in the layers \ref{fig:DWspin}. The magnetization of the ferromagnet changes the flow of spin angular momentum by exerting a torque on the flowing spins to reorient them, and therefore the flowing electrons must exert an equal and opposite torque on the ferromagnet. This torque that is applied by non-equilibrium conduction electrons onto a ferromagnet is what we will call the spin transfer torque \cite{spintransfer}.

Spin current which is a flow of spin angular momentum, is generated in addition to the charge current. The spin current normally appears in ferromagnets. However, it can also be generated also in non-magnets. The simplest method of generating a spin-polarized current in a metal is to pass the current throughout a ferromagnetic material. A common application is the GMR as mentioned before \cite{handbookspin}.

\begin{figure}[htbp]
	\centering
		\includegraphics[width=0.35\textwidth]{Figures/electron.png}
		\rule{35em}{0.5pt}
	\caption[Electron carries spin, charge and magnetic]{Electrons not only carries charge, but also spin and magnetic properties }
	\label{fig:electron}
\end{figure}


Spin polarized transport occurs naturally in any materials which have a spin imbalance between spin-up and spin-down at the fermi Level. It occurs and spin-down electrons is often nearly identical, but the states are shifted in energy with respect to each other. Fermi level is the term used to describe the top of the collection of electron energy levels at absolute zero temperature \cite{handbookspin}.

Amongst the rapidly growing variety of proposed and developed spin structures, nonlocal spin detection devices, where measurement and current excitation paths are spatially separated, have recently gained a prominent position \cite{spinz}.

\subsection{Domain Wall}

An abrupt in magnetization at the boundary of two anti-aligned domains is not a favorable condition. Domain walls form between such domains as means of minimizing the energy of the two anti-aligned domains. Domains walls are transitions layers in which the magnetization changes gradually from on magnetization to another. In other words the boundaries between regions of uniform magnetization. The gradual change prevents the large increase in exchange energy that would accompany an abrupt change in the magnetization angle. Common domain wall geometric include Bloch walls, N$\acute{e}$el walls and vortex walls\cite{spindomain}. In this study only two DW are analyzed Vortex Wall and Asymmetric Transverse Wall.

\begin{description}
  \item[Vortex Wall (VW)] \hfill \\
   In the case of Vortex wall the magnetization rotates in the place perpendicular to the domain wall, but the local magnetization is wrapped around a single vortex point. See figure \ref{fig:dw}.
   
 \item[Asymmetric Transverse Wall (ATW)] \hfill \\
 The transverse wall has a reflection symmetry about a line perpendicular to the strip axis, and a lack of symmetry about the center line of the strip. However, asymmetric transverse wall, is the absence of such symmetry, figure \ref{fig:dw}.
\end{description}

\begin{figure}[htbp]
	\centering
		\includegraphics[width=0.5\textwidth]{Figures/dw.png}
		\rule{35em}{0.5pt}
	\caption[Domain Wall VW, ATW]{Vortex Wall (VW) and Asymmetric Transverse Wall (ATW)}
	\label{fig:dw}
\end{figure}

\subsection{Spin Torque in Domain Walls}

Domain walls are the basis for various spintronics devices that uses magnetic momentums, in other words spin of electronics, the used of the spin degree of freedom. The figure \ref{fig:DWspin} illustrates a micromagnetic model of the domain wall trapped in a nanowire. The domain wall can be pushed along the wire in a controllable manner by applying an external magnetic field or by passing an electrical current through the wire \cite{dwwire}.

\begin{figure}[htbp]
	\centering
		\includegraphics[width=0.5\textwidth]{Figures/DWspin.png}
		\rule{35em}{0.5pt}
	\caption[Domain Wall nanowire]{Domain Wall in a nanowire while passing a current}
	\label{fig:DWspin}
\end{figure}

The energy of the incoming  carrier is no the only factor that determines whether or not it passes to the other side of domain wall, the spin also must be taken into account. Since each spin orientation experiences a different potential. Simulation of such properties is necessary.

Spin Torque induced domain wall motion opens up a host of possibilities for applications. The success of spintronics untimely depends on out ability to precisely  control the polarization of electrons transported within the actual thin film structure \cite{ferro}. Advances in spintronics recognized by 2007 Nobel Prize in Physics have enable over the last decade advances in computer memory, in hard drives, this is a metal based structures which utilize magnetoresistive effects to save and read data from a magnetic disk \cite{handbookspin}. An interesting application using this idea is new design for a different type memory disk drive called racetrack memory by Parkin in 2008\cite{racetrack}. The racetrack memory stores bits along a single ferromagnetic wire. To write and read information, a current is applied along the wire that moves the bits to writing or reading unit.
 
\section{Domain Wall Dynamics under Nonlocal STT}


\subsection{Theoretical Approaches}

The inclusion of STT into micromagnetics has up to now been performed with local terms that express the STT as a function of the local magnetization only. The magnetization dynamics is described by the classical Landau-Lifshitz-Gilbert (LLG) equation, expanded with a STT variable \ref{eq:llg}.

\begin{equation}  \label{eq:llg}
	\frac{\partial \vec{m}}{\partial t} = \gamma_0\vec{H}_{eff} \times \vec{m} + \alpha \vec{m} \times \frac{\partial \vec{m}}{\partial t} - \vec{T}
\end{equation}

This novel idea of incorporating spin torque into the LLG equation has itself been incorporated into a model proposed by Zhang - Li in 2004 \cite{zhang2004}. The LLG equation \ref{eq:llg} is incorporated effects of a spin-polarized current in a magnetic system, and the resulting spin transfer. They develop a form for the spin torque based on the spatial variation of the magnetization, as especially appropriate approach for domain walls. Then in 2005 the same authors Zhang - Li extended this idea working out the difference between the adiabatic and non-adiabatic torque contributions. Which lead to an even longer magnetization dynamics equation \cite{zhang} \cite{spindomain}.

\begin{equation} \label{eq:zhang}
 \frac{\partial \delta \vec{m} }{\partial t} =  D_{0}\bigtriangledown^{2} \delta \vec{m} - \frac{1}{\tau_{sd}} \delta \vec{m} \times \vec{M} - \frac{1}{\tau_{sf}}\delta \vec{m} +(\vec{\mu} \cdot\vec{\bigtriangledown} )\vec{M}
\end{equation}

The equation \ref{eq:zhang} is referred to Zhang - Li model, represents a non-adiabatic spin torque, with the presence of spin diffusion. Spin diffusion is a process by which magnetization can be exchanged spontaneously between spin. The diffusion term of the equation \ref{eq:zhang} which carriers’  drift-diffusion equation implies that the spin density does not depend solely on the local magnetization, which gives rise of nonlocal magnetics effects \cite{claudio}.

\subsection{Experiment}

We Quantitatively test the effects of spin diffusion, on real Domain walls structures, this is done by numerically solve the Zhang-Li model into micro-magnetics, using the equation \ref{eq:zhang}. The paper of \cite{zhang} initially solves analytically the diffusion equation\ref{eq:zhang}, However, ignoring the term of spin diffusion. In this numerically simulation we solve such equation using the spin diffusion term.

The sample that is considerated is a 300 nm wide and 5 nm tick NiFe soft nanostrip. This dimensions are widely used for experimental use. Two Domain walls are used a Asymmetric Transverse Wall (ATW) and a Vortex Wall (VW). ATW maps of magnetization components of non equilibrium spin accumulation under a uniform current density with $D = 0, 1$ and $10 nm^2 / ps$, see figure \ref{fig:atw}.

\begin{figure}[htbp]
	\centering
		\includegraphics[width=0.5\textwidth]{Figures/ATW.png}
		\rule{35em}{0.5pt}
	\caption[Asymmetric Transverse Wall results]{Asymmetric Transverse Wall (ATW) results}
	\label{fig:atw}
\end{figure}

Vortex Wall (VW) same as for ATW, we point out the noticeable effect of the diffusion constant around the vortex core, which is the smallest feature of the wall. Figure \ref{fig:vw}.

\begin{figure}[htbp]
	\centering
		\includegraphics[width=0.5\textwidth]{Figures/VW.png}
		\rule{35em}{0.5pt}
	\caption[Vortex Wall results]{Vortex Wall results}
	\label{fig:vw}
\end{figure}

\section{Numerical Solution}

The equation is physically realistic, However, computationally expensive \ref{Xhang}. Therefore we numerical methods to solve such equation. The numerical methods used for the solution is a the method known as Finite Differences in the Time Domain (FDTD) whose integration is done using a 4th order Runge - Kutta integration.

\subsection{Finite differences in the time domain}

The finite difference in the time domain (FDTD) method can solve complicated problems, but it is generally computationally expensive. Solutions may require a large amount of memory and computation time.  FDTD is a numerical analysis technique use for approximating solutions to the associates system of differential equations. The method belongs in the general class of grid-based differential numerical modeling methods \cite{methods}.

The FDTD method essentially uses a weighted summation of functions values at neighboring points to approximate the derivate at a particular point, in this case a point in a 3d grid. The result for each cell is based on the results from the cell and its neighbors at the previous time-frame, figure \ref{fig:fdtd}.  

\begin{figure}[htbp]
	\centering
		\includegraphics[width=0.5\textwidth]{Figures/fdtd.png}
		\rule{35em}{0.2pt}
	\caption[FDTD grid]{The result for each cell is based on evaluating the derivate cell neighbors.}
	\label{fig:fdtd}
\end{figure}

The magnetization is sampled on a uniform rectangle mesh at points $(x_0 + i\bigtriangledown_x, y_0 + j\bigtriangledown_y, z_0 + k\bigtriangledown_z)$. The computational cell is centered about the sample point with dimensions. $\bigtriangledown_x \times \bigtriangledown_y \times \bigtriangledown_z$ \cite{methods}.

Looking at the equation \ref{eq:zhang}, we need a method to calculate the first and second derivate. With the Taylor expansion we can perform such calculation. The Second order Taylor expansion readily yields expressions for the first and seconds central derivates. First and second-order derivates of the magnetization components in order to define the divergence  of the magnetization $(\nabla \cdot m)$, and the components of the exchange field $(\nabla^2m)$, respectively. The magnetization components along boundaries also need to be evaluated in order to define surface charges $(m \cdot n)$. Boundary conditions need to be incorporated in the evaluated of the effective field without loss of accuracy. 

Consider a regular, differentiable one-dimension  scalar function $f(x)$ sampled at regular intervals, a, see figure  \ref{fig:bound}. Second order Taylor expansion readily tiles expressions for the first and seconds central derivates that are widely used in numerics, namely $\frac{df}{dx} = \frac{f_{i+1} - f_{i-1}}{2a}$ and $\frac{d^2f}{dx^2} = \frac{f_{i+1} - 2f_i + f_{i-1}}{a^2}$.

\begin{figure}[htbp]
	\centering
		\includegraphics[width=0.5\textwidth]{Figures/bound.png}
		\rule{35em}{0.2pt}
	\caption[Sampled at regular intervals a, Taylor expansion]{Sampled at regular intervals a, (a) Function of inside the grid. (b) Mesh points second to closest to boundary. (c) Mesh points closet to boundary}
	\label{fig:bound}
\end{figure}

However, the numerical derivation of the structure of a simple Bloch wall using such expressions soon reveals that second order Taylor expansion ledes to restricted accuracy. Fourth order expansion as actually been found to prove much superior.  \cite{methods}

Taylor expansion of the function $f(x)$ around $x=x_i$ yields where $f^{(k)}(x_i) = f(x)$ if $k=0$

$$f(x) = \sum\limits_{k=0}^{\infty} \dfrac{(x-x_i)^k}{k!}f^{(k)}(x_i) = \sum\limits_{k=0}^{\infty} \dfrac{(x-x_i)^k}{k!}f^{(k)}$$

Applying the previous equation to nearest and next nearest neighbor to grid pint i and truncation the the 4th order yields a set of four equations

The set of linear equations provide numerical estimates for the first, second, third and fourth derivatives of $f$ at any given point $i$. The general form of the first and second derivate based on second nearest neighbors expansion reads:

\begin{align} \label{eq:nn}
f^{(1)}_i &= \dfrac{f_{i-2}-8f_{i-1} + 8f_{i+1} - f_{i+2}}{12a} \\
f^{(2)}_i &= \dfrac{f_{i-2}+16f_{i-1} -30f_{i} + 16f_{i+1} - f_{i+1}}{12a^2}
\end{align}

The equation \ref{eq:nn} for the second derivate based on second nearest neighbors expansion solves for the laplacian operator in the Zhang -Li Model equation \ref{eq:zhang}. However, points close to the edges need to be evaluated for great precession. 

\subsubsection{Boundary conditions}

Expressions such as \ref{eq:nn} are valid when the grid point becomes closet or next-to-closet to the boundary of the magnetic box. Specific accuracy preserving, expansion need to be worked out. The general principal in the present approach is to replace equations that are missing because of the lack of grid points outside the magnetic volume by equations including explicit reference to boundary conditions. \cite{methods}

Consider first a point second to closet to bound, \ref{fig:bound}-b. Grid point $i + 1$ is missing for this particular geometry. However, defining $x_R$ as the right boundary coordinate along the $x$ axis. The $f^{(1)}(x_R)$ to be know along the boundary to be replace by the derivate of Taylor's expansion

\begin{equation} \label{eq:taylor}
f^{(1)}(x) = \sum\limits_{k=0}^{\infty} \dfrac{(x-x_i)^{k-1}}{(k - 1)!}f^{(k)}(x_i)
\end{equation}

Using \ref{fig:bound}-b. $x_R - x_i = 3a/2$ becomes.

\begin{align}
\begin{bmatrix} \label{eq:matrix4}
    -2a & \dfrac{(-2a)^2}{2!} & \dfrac{-(2a)^3}{3!} & \dfrac{(-2a)^4}{4!}\\
    -a & \dfrac{(-a)^2}{2!} & \dfrac{(-a)^3}{3!} & \dfrac{(-a)^4}{4!}\\
    a & \dfrac{(a)^2}{2!} & \dfrac{(a)^3}{3!} & \dfrac{(a)^4}{4!}\\
    2a & \dfrac{(2a)^2}{2!} & \dfrac{(2a)^3}{3!} & \dfrac{(2a)^4}{4!}
\end{bmatrix}
\begin{bmatrix}
    f_i^{(1)}  \vphantom{ \dfrac{d^4}{d} }\\
    f_i^{(2)}  \vphantom{ \dfrac{d^4}{d} } \\
    f_i^{(3)}  \vphantom{ \dfrac{d^4}{d} } \\
    f_i^{(4)}  \vphantom{ \dfrac{d^4}{d} }
\end{bmatrix}
=
\begin{bmatrix}
    f_{i-2} - f_{i}    \vphantom{ \dfrac{d^4}{d}} \\
    f_{i-1} - f_{i}    \vphantom{ \dfrac{d^4}{d}} \\
    f_{i+1} - f_{i}    \vphantom{ \dfrac{d^4}{d}} \\
    f^{(1)}(x_R)   \vphantom{ \dfrac{d^4}{d}}
\end{bmatrix}
\end{align}

Similarly, for a point closet to boundary, reference \ref{fig:bound}-c, grid points i + 1 and i + 2 are missing. The two first equation of ... need now to be replaced by a single equation, whilst the two remaining equations need to be truncated to the third order. For the geometry illustrated in \ref{fig:bound}-c, the minimal set of equations now reads.

\begin{align} \label{eq:matrix3}
\begin{bmatrix}
    -2a & \dfrac{(-2a)^2}{2!} & \dfrac{-(2a)^3}{3!}\\
    -a & \dfrac{(-a)^2}{2!} & \dfrac{(-a)^3}{3!}\\
    1 & \dfrac{(+a)}{2} & \dfrac{(+a/2)^3}{2!}
\end{bmatrix}
\begin{bmatrix}
    f_i^{(1)}  \vphantom{ \dfrac{d^4}{d} }\\
    f_i^{(2)}  \vphantom{ \dfrac{d^4}{d} } \\
    f_i^{(3)}  \vphantom{ \dfrac{d^4}{d} }
\end{bmatrix}
=
\begin{bmatrix}
    f_{i-2} - f_{i}    \vphantom{ \dfrac{d^4}{d}} \\
    f_{i-1} - f_{i}    \vphantom{ \dfrac{d^4}{d}} \\
    f^{(1)}(x_R)   \vphantom{ \dfrac{d^4}{d}}
\end{bmatrix}
\end{align}

In both cases, and second derivatives and fully determined provided $f^{(1)}(x_R)$ be known along the boundary. For further reference \cite{methods}. The implementation to solve the laplacian with boundaries conditions check chapter 4.

The main advantages of the finite difference approach are ease of implementation, simplicity of meshing, efficient evaluation of the magnetization energy, and the accessibility of higher order methods. the main disadvantage of this approach is the sampling curved boundaries with a rectangular mesh, resulting in some what discrete  approximation, which ca produce significant error in the computation.

\subsection{Fourth order Runge and Kutta method}

 Modern numerical algorithms for the solution of ordinary differential equations are based on the method of the Taylor series. Algorithm such as the Runge-Kutta method are constructed so they give an expression depending of the parameter $(h)$, in other works the step as an approximate solution of the first terms of the Taylor series. \cite{ufdtd}
The method can accurately solve a wide range of problems, but it is generally computationally expensive. Solutions require large amount of memory and computational time.

There exist several other computational numeric methods to solver such equations, methods such as the Euler integrator, the Midpoint Method and the Runge-Kutta fourth order (RK4) integrator method can solve differential equations. However, they differ in the numerically approximation and computation time. The RK4 is used for this simulation because its numerically more accurate when compared to the others methods.

The RK4 method differs widely from the Euler method and the Midpoint method. The Euler method is the simplest, the derivative at the starting point of each interval is extrapolated to find the next function value, see figure \ref{fig:euler}. Euler method only has first order accuracy while the RK4 its fourth order integrator \cite{numerical}.

\begin{figure}[htbp]
	\centering
		\includegraphics[width=0.55\textwidth]{Figures/euler.png}
		\rule{35em}{0.4pt}
	\caption[Euler Method]{Euler Method, Is the simplest approximate to solver differential equation or numerically solve equations.}
	\label{fig:euler}
\end{figure}

RK4 goes as follows:

\begin{equation} \label{eq:rk4}
y_{n+1} = y_{n} + 1/6 K_{1} + 1/3 K_{2} +1/3 K_{3} + 1/6 K_{4}
\end{equation}
where
\begin{equation}
\begin{split} \label{eq:rksplit}
K_{1} &= h \dot f(x_{n}, y_{n}) \\
K_{2} &= h \dot f(x_{n} + h/2, y_{n} + k_{1}/2) \\
K_{3} &= h \dot f(x_{n} + h/2, y_{n} + k_{2}/2) \\
K_{4} &= h \dot f(x_{n} + h, y_{n} + k_{3})
\end{split}
\end{equation}

As the equations shows, each step, the derivative is evaluated four times, once at the initial point, twice at trial midpoints, and once at a trial endpoint. From these four values, the final value is calculated, just like the equation \ref{eq:kg4}.

\begin{figure}[htbp]
	\centering
		\includegraphics[width=0.45\textwidth]{Figures/rk4.png}
		\rule{35em}{0.5pt}
	\caption[Fourth-order Runge and Kutta Method]{Fourth-order Runge and Kutta method, Each step the derivative is evaluated four times. }
	\label{fig:kutta}
\end{figure}

\subsection{Effective Beta}

We show that and effective non-adiabatic parameter $\beta_{diff}$ dependent on the DW structure, provides 
\begin{equation} \label{eq:beta}
(\beta \upsilon)^{*} = \frac{\delta\vec{m} \cdot \partial_x \vec{m}}{\tau_{sd} \| \partial_x \vec{M} \|^2}
\end{equation}


-----fill final calculation.


The procedure is (i) compute the non-equilibrium spin density $\delta \vec{m}$ with the DW at rest, solve equation \ref{eq:zhang} to convergence or directly its time-independent version. (ii) compute the $\beta_{diff}$ distribution from equation \ref{eq:beta} and finally compute $\beta_{diff}$ by averaging with the $| \partial_x \vec{m}|^2$ weight function.

%The basic structure is computational solve rungge and kutta of for other.

\vspace{3.5em}

In conclusion, a simultaneous solution of the diffusive Zhang and Li model \ref{eq:zhang} has uncovered a qualitatively new feature of the spin-transfer torque effect in the presence of spin diffusion. Namely the dependence of the steady-state DW velocity on DW structure \cite{claudio}. In summarize, we quantitatively test the effects of spin diffusion, on real Domain walls structures for ATW and VW. This is done by numerically solve the Zhang-Li model into micro-magnetics. The numerical methods used to solve such model as mentioned is the FDTD on a 3D cell grid with whose integration is done using RK4.



%Ind addition, these results offers a starting point to study multilayer structures like spin-value nano strips, where the understanding of the observed increased efficecnctiy of SST to drive DW's still remains elusive. 



