% Chapter Template

\chapter{Introduction to Domain Wall Dynamics and a Implementation with CUDA} % Main chapter title

\label{Introduction to DW Dynamics} % Change X to a consecutive number; for referencing this chapter elsewhere, use \ref{ChapterX}

\lhead{Chapter 3. \emph{Introduction to DW Dynamics}}% Change X to a consecutive number; this is for the header on each page - perhaps a shortened title

%----------------------------------------------------------------------------------------
%	SECTION 1
%----------------------------------------------------------------------------------------
This chapter is a overview of the theory and experiments behind the study of Domain Wall Dynamics under Nonlocal Spin-Transfer Torque. This is a quantitatively test the effects of spin-diffusion, on real Domain Wall (DW) structures, by numerically implementing the Zhang-LI model into a NiFe soft nanostrip. The numericall implementation takes advantage of the highly parallel process capabilities of the GPU. The numerical method used for the solution is a the method known as Finite Differences in the Time Domain (FDTD) whose integration is done using a 4th order Runge - Kutta integration.


\section{Theory}

 \cite{claudio}

Contrarily to charge, spin accumulate in metals, The associated diffusion current flows in all directions, giving rise to nonlocal effects, Beyond transport properties, conduction electrons spin resonance and spin pumping provide further testimonies for non-locality in spin transport. These works all refer  to samples consisting in piecewise uniform layers or blocks, magnetic or not. Of special significance to the present work in the non-collinear geometry where a spin current with polarization transverse to the magnetization exists, whose absorption in the vicinity of the surface of a magnetic layer creates a torque on the magnetization, known as spin transfer torque (SFF),  \cite{claudio}

Spintronics is a new type of electronics that exploits the spin degree of freedom of an electron in addition to its charge. \cite{spinz}

\cite{ferro}

\subsection{Domain Wall}

An abrupt in magnetization at the boundary of two anti-aligned domains is not a favorable condition. Domain walls form between such domains as means of minimizing the energy of the two anti-aligned domains. Domains walls are transitions layers in which the magnetization changes gradually from on magnetization to another. In other words the boundaries between regions of uniform magnetization.  The gradual change prevents the large increase in exchange energy that would accompany an abrupt change in the magnetization angle. Common domain wall geometrias include Bloch walls, N$\acute{e}$el walls and vortex walls. In the case of Vortex wall the magnetization rotates in the place perpendicular to the domain wall, but the local magnetization is wrapped around a single vortex point \cite{spindomain}. This can seen in figure \label{eq:kg4}.

\begin{figure}[htbp]
	\centering
		\includegraphics[width=0.5\textwidth]{Figures/vortex.png}
		\rule{35em}{0.5pt}
	\caption[Domain Wall - Vortex]{Domain Wall - Vortex}
	\label{fig:vortex}
\end{figure}

Domain walls are the basis for various spintronics devices that uses magnetic momentums, in other words spin of electronics. the used of the spin degree of freedom. With this it is expected that electronics technology and devices will be faster, compacter and more energy-saving. A interesting application using this idea is new design for a different type memory disk drive called racetrack memory by Parkin in 2008\cite{racetrack}

Spin-transfer torque is a torque that exerts on a magnetization by conduction electron spins, in other words the angular momentum transferred from spins to magnetic  moment \cite{zhang}.
This has simulated research into domain wall (DW) dynamics, particularly those resulting from interactions with current passing through the DW via the phenomenon of spin momentum transfer (SMT) \cite{handbookspin}

The study spin-diffuse effect within a continuously variable magnetization distribution, integrating with micromagenectis with diffuse model of Zhang and LI \cite{claudio}

We Quantitatively test the effects of spin diffusion, on real Domain walls structures, this is done by numerically solve the Zhang-Li model \cite{zhang} into micro-magnetics.
The Zhang Li model refers to:

which is the following equation.



\section{Experiment}

Base on the work of Dr. Claudio \cite{claudio}

At first we investigate the steady-sate velocity regime of DWs in NiFe soft nanostrips. applying current densities similar to those reported in experiments. The results that we are going to obtain

Experimentally measured spin-diffusion parameters are used, we want to the solution of.

\begin{equation}
 \frac{\partial \delta \vec{m} }{\partial t} =  D\bigtriangleup \delta \vec{m} + \frac{1}{\tau_{sd}} \vec{m} \times \delta  \vec{m} - \frac{1}{\tau_{sf}}\delta \vec{m} - u \partial_{x}  \vec{m}
\end{equation}

The sample that is considerate is a 300 nm wide and 5 nm tick NiFe soft nanostrip. This dimensions are widely used for experimental use.

Therefore, a simultaneous solution of the diffusive Zhang and Li model together with the magnetization dynamics equation has uncovered a qualitatively new feature of the spin-transfer torque effect in the presence of spin diffusion.


Advances in spintronics recognized by 2007 Nobel Prize in Physics have enable over the last decade advances in computer memory, in hard drives, this is a metal based structures which utilize magnetoresisite effects to save and read data from a magnetic disk. \cite{handbookspin}

Some application include racetrack technology by the IBM fellow scientific Parkin \cite{racetrack}

Base on this study numeric applications have been unfold.


%-----------------------------------
%	SUBSECTION 1
%-----------------------------------
\section{Numerical Methods}


%The implementation of the GPU of Dr. Claudio is based on launching several kernels into a single GPU node.

\subsection{Laplacian}


\cite{landaverde}

The differential evaluation in one-dimension,
The Second order Taylor expansion readily yields expressions for the first and seconds central derivates

$$ \dfrac{df}{dx} = \dfrac{f_{i+1} - f_{i-1} }{2a}$$

and

$$ \dfrac{d^{2}f}{dx^{2}} = \dfrac{f_{i+1} - 2f_{i}+f_{i-1} }{a^2}$$

\begin{figure}[htbp]
	\centering
		\includegraphics[width=0.4\textwidth]{Figures/taylor.png}
		\rule{35em}{0.2pt}
	\caption[Sampled at regular intervals a, Taylor expansion]{Sampled at regular intervals a}
	\label{fig:taylor}
\end{figure}

Taylor expasion of the function $f(x)$ around $x=x_i$ yields where $f^{(k)}(x_i) = f(x)$ if $k=0$

$$f(x) = \sum\limits_{k=0}^{\infty} \dfrac{(x-x_i)^k}{k!}f^{(k)}(x_i) = \sum\limits_{k=0}^{\infty} \dfrac{(x-x_i)^k}{k!}f^{(k)}$$

Applying the previous equation to nearest and next neares neigborar to grid pint i and tructation tht the 4th order yields a set of four equations:

\begin{align}
\begin{bmatrix}
    -2a & \dfrac{(-2a)^2}{2!} & \dfrac{-(2a)^3}{3!} & \dfrac{(-2a)^4}{4!}\\
    -a & \dfrac{(-a)^2}{2!} & \dfrac{(-a)^3}{3!} & \dfrac{(-a)^4}{4!}\\
    a & \dfrac{(a)^2}{2!} & \dfrac{(a)^3}{3!} & \dfrac{(a)^4}{4!}\\
    2a & \dfrac{(2a)^2}{2!} & \dfrac{(2a)^3}{3!} & \dfrac{(2a)^4}{4!}
\end{bmatrix}
\begin{bmatrix}
    f_i^{(1)}  \vphantom{ \dfrac{d^4}{d} }\\
    f_i^{(2)}  \vphantom{ \dfrac{d^4}{d} } \\
    f_i^{(3)}  \vphantom{ \dfrac{d^4}{d} } \\
    f_i^{(4)}  \vphantom{ \dfrac{d^4}{d} }
\end{bmatrix}
=
\begin{bmatrix}
    f_{i-2} - f_{i}    \vphantom{ \dfrac{d^4}{d}} \\
    f_{i-1} - f_{i}    \vphantom{ \dfrac{d^4}{d}} \\
    f_{i+1} - f_{i}    \vphantom{ \dfrac{d^4}{d}} \\
    f_{i+2} - f_{i}    \vphantom{ \dfrac{d^4}{d}}
\end{bmatrix}
\end{align}

The set of linear equations provide numerical estimates for the first, second, third and fourth derivatives of $f$ at any given point $i$.

The general form of the first and second derivate based on second nearest neighbors expansion reads:

\begin{align*}
f^{(1)}_i &= \dfrac{f_{i-2}-8f_{i-1} + 8f_{i+1} - f_{i+2}}{12a} \\
f^{(2)}_i &= \dfrac{f_{i-2}+16f_{i-1} -30f_{i} + 16f_{i+1} - f_{i+1}}{12a^2}
\end{align*}


\begin{align}
\begin{bmatrix}
    -2a & \dfrac{(-2a)^2}{2!} & \dfrac{-(2a)^3}{3!} & \dfrac{(-2a)^4}{4!}\\
    -a & \dfrac{(-a)^2}{2!} & \dfrac{(-a)^3}{3!} & \dfrac{(-a)^4}{4!}\\
    a & \dfrac{(a)^2}{2!} & \dfrac{(a)^3}{3!} & \dfrac{(a)^4}{4!}\\
    1 & \dfrac{(3a)}{2} & \dfrac{(3a/2)^3}{3!} & \dfrac{(3a/2)^4}{4!}
\end{bmatrix}
\begin{bmatrix}
    f_i^{(1)}  \vphantom{ \dfrac{d^4}{d} }\\
    f_i^{(2)}  \vphantom{ \dfrac{d^4}{d} } \\
    f_i^{(3)}  \vphantom{ \dfrac{d^4}{d} } \\
    f_i^{(4)}  \vphantom{ \dfrac{d^4}{d} }
\end{bmatrix}
=
\begin{bmatrix}
    f_{i-2} - f_{i}    \vphantom{ \dfrac{d^4}{d}} \\
    f_{i-1} - f_{i}    \vphantom{ \dfrac{d^4}{d}} \\
    f_{i+1} - f_{i}    \vphantom{ \dfrac{d^4}{d}} \\
    f_{i+2}(x_R)   \vphantom{ \dfrac{d^4}{d}}
\end{bmatrix}
\end{align}


\cite{methods}

\subsection{Finite Differences in the Time Domain}

Modern numerical algorithms for the solution of ordinary differential equations are also base on the method of the Taylor series. Each algorithm such as Runge-Kutta method are constructed so they give an expression depending of the parameter $(h)$, in other works the step as an approximate solution of the first terms of the Taylor series. \cite{ufdtd}
The method can acurately tackle a wide range of problems as weel as can solve complicated problem, but it is generally computacinaly expensive. Solutions requiere large amount of memory and computacional time. 

The FDTD method employs finite differences as approximations to both the spatial and temporal derivates that appear in Maxwell's equations


\subsection{Fourth order Runge and Kutta method}

There exist several computational numeric methods to solver such equations, methods like Euler, Midpoint Method and Runge-Kutta integrator method can solve this type of equations. The RG4 this method is used for the simulation because its numerically more accurate when compared to the others.

The RG4 method differs widely from the Euler method and the Midpoint method. The euler method is the simplest, the derivative at the starting point of each interval is extrapolated to find the next function value, see figure \ref{fig:euler}. Euler method only has first order accuracy while the RG4 its fourth order integrator.

\begin{figure}[htbp]
	\centering
		\includegraphics[width=0.7\textwidth]{Figures/euler.png}
		\rule{35em}{0.5pt}
	\caption[Euler Method]{Euler Method, Is the simplest approximate to solver differential equation or numerically solve equations.}
	\label{fig:euler}
\end{figure}

RK4 goes as follows:

\begin{equation} \label{eq:kg4}
y_{n+1} = y_{n} + 1/6 K_{1} + 1/3 K_{2} +1/3 K_{3} + 1/6 K_{4}
\end{equation}
where

\begin{align*}
K_{1} &= h \dot f(x_{n}, y_{n}) \\
K_{2} &= h \dot f(x_{n} + h/2, y_{n} + k_{1}/2) \\
K_{3} &= h \dot f(x_{n} + h/2, y_{n} + k_{2}/2) \\
K_{4} &= h \dot f(x_{n} + h, y_{n} + k_{3})
\end{align*}

As the equations shows, each step, the derivative is evaluated four times, once at the initial point, twice at trial midpoints, and once at a trial endpoint. From these four values, the final value is calculated, just like the following equation \ref{eq:kg4}

\begin{figure}[htbp]
	\centering
		\includegraphics[width=0.5\textwidth]{Figures/rk4.png}
		\rule{35em}{0.5pt}
	\caption[Fourth-order Runge and Kutta Method]{Fourth-order Runge and Kutta method, Each step the derivative is evaluated four times. }
	\label{fig:kutta}
\end{figure}

\cite{numerical}

%The basic structure is computational solve rungge and kutta of for other.
