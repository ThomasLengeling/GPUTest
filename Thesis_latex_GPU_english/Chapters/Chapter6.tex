% Chapter Template

\chapter{Conclusions and future work} % Main chapter title

\label{Conclusions and future work} % Change X to a consecutive number; for referencing this chapter elsewhere, use \ref{ChapterX}

\lhead{Chapter 6. \emph{Conclusions and future work}} % Change X to a consecutive number; this is for the header on each page - perhaps a shortened title

%----------------------------------------------------------------------------------------
%	SECTION 1
%----------------------------------------------------------------------------------------


GPUs definitely have a place in the world of computational physics, their use allows to do the same work with less energy and more science with less resources. They make HPC computing clusters affordable for small research groups. The true limit test of this new technology will be if it is actually used to advance new science. In the field of computational physics studies that do push the barrier of what is computationally feasible, from speedups of 1.5x to 20x using GPUs\cite{applications}.

Acceptance has been slow due to many factors, GPUs are sometimes seen as a fad or a niche, the specialized skill set and effort required for GPU programming along with the risk of spending money to setup a GPU cluster, does raise a concern for productivity and viability of this technology. Adopting this technology requires abandoning legacy codes and smart optimizations that have been developed over the years. A wrong choice may result in wasted time and effort.

What is certain is at the moment, is the overall direction of the industry towards higher parallelism, as single threaded performance has reached a local limit, all types of processors are seeking more performance out of parallelism. This means that a large portion of the work needed to parallelize a code for a certain parallel architecture will most probably be applicable to another parallel architecture as well. From the literature and my experiences, one can observe that in order to achieve good results in programming with GPUs it is necessary to take a Heterogeneous approach to coding. That is adopting multi-threaded CPUs and concurrent GPU type algorithms.

Spintronics. In particular we are involved in designing new magnetic materials for spin-devices and modeling and understanding of spin-transport at molecular and atomic scale. By computer simulation is possible to predict their output. Furthermore prove the theoretical experiments.

In the simu


The current thread is to push the hardware capabilities and performance along with Mooers' Law, despite these issues there are some trends in the hardware industry that will make working with GPU easier and more widespread within a HPC context:

\begin{description}
  \item[3D Memory] \hfill \\
 Stacks DRAM chips into dense modules with wide interfaces, and brings them inside the same package as the GPU. This lets GPUs get data from memory more quickly – boosting throughput and efficiency – allowing us to build more compact GPUs that put more power into smaller devices. The result: several times greater bandwidth, more than twice the memory capacity and quadrupled energy efficiency.
  
  \item[NVLink] \hfill \\
 Today’s computers are constrained by the speed at which data can move between the CPU and GPU. NVLink puts a fatter pipe between the CPU and GPU, allowing data to flow at more than 80GB per second, compared to the 16GB per second available now.
 
 \item[Pascal Module] \hfill \\ 
  NVIDIA has designed a module to house Pascal GPUs with NVLink. At one-third the size of the standard boards used today, they’ll put the power of GPUs into more compact form factors than ever before.
  
   \item[Mobile and embedded Devices] \hfill \\ 
   Kepler
   Erista Maxwell GPU
  
   \item[Cloud Computing] \hfill \\ 

  \end{description}


To conclude, I offer my personal perspective on GPU computing. I think the importance of using accelerator hardware is an economic and environmental issue. The environmental aspect of doing computing is often overlooked, but an ever increasing important one. As heavy computer users we will have to take responsibility for our electricity use. The benefit of less energy use is clear.

