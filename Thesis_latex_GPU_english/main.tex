
%Thomas Sanchez Lengeling Bachelor's Thesis

%----------------------------------------------------------------------------------------
%	PACKAGES AND OTHER DOCUMENT CONFIGURATIONS
%----------------------------------------------------------------------------------------

\documentclass[11pt, oneside, a4paper]{Thesis} % The default font size and one-sided printing (no margin offsets)

\graphicspath{{Pictures/}} % Specifies the directory where pictures are stored

\usepackage[square, numbers, comma, sort&compress]{natbib} % Use the natbib reference package - read up on this to edit the reference style; if you want text (e.g. Smith et al., 2012) for the in-text references (instead of numbers), remove 'numbers'


\usepackage[font={small}]{caption}

\usepackage[]{algorithm2e}

\usepackage{listings} 
\usepackage{xcolor}
\usepackage{url}

%\lstset{
%  language=C,                % choose the language of the code
%  numbers=left,                   % where to put the line-numbers
%  stepnumber=1,                   % the step between two line-numbers.        
%  numbersep=5pt,                  % how far the line-numbers are from the code
%  backgroundcolor=\color{white},  % choose the background color. You must add   \usepackage{color}
%  showspaces=false,               % show spaces adding particular underscores
%  showstringspaces=false,         % underline spaces within strings
%  showtabs=false,                 % show tabs within strings adding particular underscores
%  tabsize=2,                      % sets default tabsize to 2 spaces
%  captionpos=b,                   % sets the caption-position to bottom
%  breaklines=true,                % sets automatic line breaking
%  breakatwhitespace=true,         % sets if automatic breaks should only happen at whitespace
%  title=\lstname,                 % show the filename of files included with\lstinputlisting;
%} 

\lstset{
    frame=tb, % draw a frame at the top and bottom of the code block
    tabsize=4, % tab space width
    showstringspaces=false, % don't mark spaces in strings
   %numbers=left, % display line numbers on the left
    commentstyle=\color{green}, % comment color
    keywordstyle=\color{blue}, % keyword color
    stringstyle=\color{red} % string color
}




\hypersetup{urlcolor=blue, colorlinks=true} % Colors hyperlinks in blue - change to black if annoying
\title{\ttitle} % Defines the thesis title - don't touch this

%\usepackage[T1]{fontenc}
%\usepackage[latin1]{inputenc}

\newcommand*{\listf}{\fontfamily{lmtt}\selectfont}

\begin{document}



\frontmatter % Use roman page numbering style (i, ii, iii, iv...) for the pre-content pages

\setstretch{1.3} % Line spacing of 1.3

% Define the page headers using the FancyHdr package and set up for one-sided printing
\fancyhead{} % Clears all page headers and footers
\rhead{\thepage} % Sets the right side header to show the page number
\lhead{} % Clears the left side page header

\pagestyle{fancy} % Finally, use the "fancy" page style to implement the FancyHdr headers

\newlength\dunder
\settowidth{\dunder}{\_}

\newcommand{\HRule}{\rule{\linewidth}{0.5mm}} % New command to make the lines in the title page
\newcommand{\twoline}{\rule{2\dunder}{0.4pt}}

% PDF meta-data
\hypersetup{pdftitle={\ttitle}}
\hypersetup{pdfsubject=\subjectname}
\hypersetup{pdfauthor=\authornames}
\hypersetup{pdfkeywords=\keywordnames}

%----------------------------------------------------------------------------------------
%	TITLE PAGE
%----------------------------------------------------------------------------------------
\begin{titlepage}
\begin{center}

\begin{figure}[htbp]
	\centering
		\includegraphics[width=0.30\textwidth]{Figures/uni.jpg}
\end{figure}

\textsc{\LARGE Universidad de Guanajuato}\\[0.5cm] % University name
\textsc{\Large Campus Irapuato Salamanca }\\
\textsc{\Large Division de Ingenierias } \\[1.5cm]


\HRule \\[0.4cm] % Horizontal line
{\huge \bfseries \ttitle}\\[0.4cm] % Thesis title
\HRule \\[1.1cm] % Horizontal line

\textsc{\Large Tesis Profesional }\\
\large \textit{Para obtener el grado en Licenciatura en Ingeniería en Sistemas Computacionales}\\[1.0cm] %


\begin{minipage}{0.4\textwidth}
\begin{flushleft} \large
\emph{Autor:}\\
{\authornames} % Author name - remove the \href bracket to remove the link
\end{flushleft}
\end{minipage}
\begin{minipage}{0.4\textwidth}
\begin{flushright} \large
\emph{Director:} \\
{\supname}\\ % Supervisor name - remove the \href bracket to remove the link
\emph{Co-Director:}\\
{Dra. Mar\'ia Susana \'Avila Garc\'ia}
\end{flushright}
\end{minipage}\\[0.8cm]

\begin{minipage}{0.83\textwidth}
\begin{flushleft} \large
\emph{Sinodales:}\\
{\examineray \hspace{2.5em} \examinerav } % Author name - remove the \href bracket to remove the link
\end{flushleft}
\end{minipage}\\[1cm]

Guanajuato, Mexico

%\groupname\\\deptname\\[2cm] % Research group name and department name

{\large \today}\\[3cm] % Date
%\includegraphics{Logo} % University/department logo - uncomment to place it

\vfill
\end{center}

\end{titlepage}

%----------------------------------------------------------------------------------------
%	ABSTRACT PAGE
%----------------------------------------------------------------------------------------

\addtotoc{Abstract} % Add the "Abstract" page entry to the Contents

\abstract{\addtocontents{toc}{\vspace{1em}} % Add a gap in the Contents, for aesthetics

This work is an exploration of the role that Graphical Processing Units, also known as GPUs, can play in the acceleration of physical simulations. In particular, in the research of spintronic effects such as the dynamics of domain walls under nonlocal spin-transfer torque. 
Our study is relevant because it allows researchers to quantitatively test some of the effects of a phenomenon known as spin-diffusion on magnetic configurations at the nanoscale. Some of such configurations are known as domain walls. These magnetic configurations can be observed experimentally in NiFe soft nanostripes but they are really complicated to produce and image experimentally. Due to this, we use the massively parallel capabilities of a single GPU to numerically solve a mathematical equation, known as the Zhang-Li Model. As a consequence of our implementation, we have observed a 8.0x speed-up in the solution of the equation. This speedup is obtained when we compare the time needed to obtain the result of a simulation in a GPU with that of a simulation with the same input parameters in a conventional processor e.g. Intel Xeon. The numerical method used for the solution is a the method known as Finite Differences in the Time Domain (FDTD) whose integration is done using a 4th order integrator.
}

\clearpage % Start a new page

%----------------------------------------------------------------------------------------
%	ACKNOWLEDGEMENTS
%----------------------------------------------------------------------------------------

\setstretch{1.3} % Reset the line-spacing to 1.3 for body text (if it has changed)

\acknowledgements{\addtocontents{toc}{\vspace{1.2em}} % Add a gap in the Contents, for aesthetics

To my advisor David Claudio for being supportive and flexible throughout the thesis endeavor; he allowed me to continue learning GPU though various experiences.

I want to thank my family, especially my parents, Manuel and Martha, and my two brothers Benjamin and Gabriel, as well as everyone that help me through the process. Karen encouraged me to finish fast and stop playing DOTA.

All my bachelor colleagues: we survived together through the courses, projects and work, and all the friends that I made along the way.
Finally, to the University of Guanajuato for the support and assistance in the process. I had numerous experiences at the University that will allowed me to continue to improve myself.

}
\clearpage % Start a new page

%----------------------------------------------------------------------------------------
%	LIST OF CONTENTS/FIGURES/TABLES PAGES
%----------------------------------------------------------------------------------------

\pagestyle{fancy} % The page style headers have been "empty" all this time, now use the "fancy" headers as defined before to bring them back

\lhead{\emph{Contents}} % Set the left side page header to "Contents"
\tableofcontents % Write out the Table of Contents

\lhead{\emph{List of Figures}} % Set the left side page header to "List of Figures"
\listoffigures % Write out the List of Figures



%----------------------------------------------------------------------------------------
%	INTRODUCTION
%----------------------------------------------------------------------------------------

\setstretch{1.3}

\introductions

\bigskip

Commodity graphics processing units (GPUs) are becoming increasingly popular to accelerate scientific applications due to their low cost and potential for high performance when compared with central processing units (CPUs). A large number of contemporary problems and scientific research are being benefited from this new technology. There has been considerable progress in implementing the hardware and the supporting infrastructure for GPUs programming and streaming architectures. This thesis is a study of heterogenous computing using NVIDIA's GPU applied to computational physics.

The first chapter is a overview of heterogeneous architecture programming with NVIDIA's GPUs using NVIDIA's programming framework, Compute Unified Device Architecture (CUDA). The second chapter is the study of the phenomenon known as spin-diffusion on magnetic configurations at the nanoscale. Some of such configurations are known as domain walls. These magnetic configurations can be observed experimentally in NiFe soft nanostripes. Due to this, we use the massively parallel capabilities of a single NVIDIA GPU to numerically solve a mathematical equation, known as the Zhang-Li Model. The numerical implementation is done by using NVIDIA's CUDA platform, which is explained in chapter 3. In addition, the numerical solution is a the method known as Finite Differences in the Time Domain (FDTD) whose integration is done using a 4th order Runge-Kutta integration. The fourth chapter focuses on optimization techniques and practices, to gain the most performance out of the hardware capabilities. The fifth chapter are the results collected by applying optimization techniques to the initial CUDA code. Techniques applyed such as concurrent kernels, shared memory, branching and occupancy. The outcome is compared by launching the code on-to several GPUs nodes using the supercomputer Piritakua. Finally, the last chapter of the thesis is a conclusion of the work and future research.


\clearpage



%----------------------------------------------------------------------------------------
%	THESIS CONTENT - CHAPTERS
%----------------------------------------------------------------------------------------

\mainmatter % Begin numeric (1,2,3...) page numbering

\pagestyle{fancy} % Return the page headers back to the "fancy" style

% Include the chapters of the thesis as separate files from the Chapters folder
% Uncomment the lines as you write the chapters




\chapter{Heterogeneous Computing} % Main chapter title

\label{Heterogeneous Computing} % Change X to a consecutive number; for referencing this chapter elsewhere, use \ref{ChapterX}

\lhead{Chapter 1. \emph{Heterogeneous Computing}} % Change X to a consecutive number; this is for the header on each page - perhaps a shortened title

%----------------------------------------------------------------------------------------
%	SECTION 1
%----------------------------------------------------------------------------------------

Heterogeneous computing refers a system that combines several processor types to gain more performance. Typically using a single or multi-core computer processing units (CPUs) and a  graphics processing units (GPUs).
Frequently GPUs are know for 3D graphics rendering, video games and video editing, but GPUs are becoming increasingly popular for accelerating computing applications and scientific research due to their low price, high performance and relatively low energy consumption per FLOPS (floating point operations per second) when compared with the CPUs. This chapter provides an overview of GPUs within the High Performance Computing (HPC) context, their advantages and disadvantages and how they can be integrated in to a scientific software and research.

%per watt better than cpu's


\section{Motivation}


The GPU has been essential part of personal computer since the early use. Over the course of 30 years the graphics architecture has evolve form drawing a simple 3D scene to be able to program each part of the GPU graphics pipeline. Their role became more important in the 90s with the first-person shooting video game DOOM by id Software. The demanding video game industry has brought year by year more realistic 3D graphics. Consequently new innovated hardware capabilities has been developed to increase the graphics pipeline and the render output. This lead to a more sophisticated programming environment with a massive parallel capabilities.

The fixed graphics pipeline (fixed functions on the GPU) was introduced in the early 90s, allowed various customization of the rendering process. However only allowed some modifications of the GPU output. Specific adjustment were extremely complicated did not allow custom algorithms. In 2001 NVIDIA and ATI (AMD) introduced the first programmability to the graphics pipeline. Which could control millions pixels and vertex output in a single frame, moreover it out-performed the CPU in rending video. In addition graphics shift from the CPU to the GPU. This was the beginning of GPU parallel capabilities.

At first the GPUs where only used for general-purpose computing like computer graphics, but in-till resent years the GPU has been used to accelerate scientific research, analytics, engineering, robotics and consumer applications.(GPGPU)\cite{physicsgpu}.

GPUs are attractive for certain type of scientific computation as they offer potential seed-up of multi-processors devices with the added advantages of being low cost, low maintenance, energy efficient, and relative simple to program. Many algorithms in applied physics are using GPUs to improve their performance over the CPU. Some examples are Euler Solver 16x seed-up (add Reference seed-up).

In any case, for a given simulation a compromise between speed and accuracy is always made. The current tendency of the CPU relies on increases the clock seeped, decreasing the size of transistor and finally adding more cores per unit and be able to work and a parallel manner, because of the there are some limitations\cite{quantitative}

\begin{description}
  \item[Power Wall] \hfill \\
  The CPUs single core has not gone beyond the 4GHz barrier, a paradigm shift from a single core to a multi-core CPUs, also  the power use of CPUs is very high per Watt. The figure \ref{fig:gpu_cpu_s} shows the comparison of performance between the GPU and CPU.

\begin{figure}[htbp]
	\centering
		\includegraphics[width=0.70\textwidth]{Figures/GPU_CPU_s.png}
		\rule{35em}{0.5pt}
	\caption[GPU and CPU]{GPU and CPU peak performance in gigaflops}
	\label{fig:gpu_cpu_s}
\end{figure}


  \item[Memory Wall] \hfill \\
  This refers to the growing disparity of speed between CPU  and the memory outside the CPU chip. Some applications have become memory bound, that is to say computing time is bounded by the transfer memory between the CPU and all the hardware devices connected to the CPU, commonly to the Peripheral Component Interconnect (PCI) chip. In conclusion the computing time is bounded by the memory not by the time calculations done on the CPU.

  \item[Parallelism Wall] \hfill \\
  This indicates a law that indicates the number of parallel processes. The number N parallel processes is never ideal and always depends on the problem.  The seed-up can be described by Amdahl's Law in terms of the fraction of parallelized work (f). \cite{quantitative}.

  $$speedup \leq \frac{N}{f + N(1-f)}$$


\end{description}

The current paradigm of using CPUs for computing growth is unsustainable. In 2012, Japan among the countries with elite supercomputers, builded the machine "K Computer", with 705,024 multi-core CPUs, it can achieve 11.3 petaflops ($10^5$ flops). Furthermore the computer is one of the most power efficient supercomputer in the world with a total of 12.66 megawatts (MW), in other words 830 Mflops/watt. This is this is enough to power a small town of 10,000 homes. If the current thread of power use continues, the next supercomputer would require 200 MW of power, this would require a nuclear power reactor to run it.\cite{whatexascale}. However in 2013 Oak Ridge Nacional Laboratory (U.S) built a supercomputer that combines CPUs and GPUs, the Titan. It can archive an astonish 24 petaflops theoretical peak. Moreover with a power consumption of 8.2 MW. Combining CPUs GPUs is possible to built supercomputers with a higher performance and lower power consumption. \cite{titan}


As said the GPU exceeds the CPU in calculations per second FLOPS with a low energy consumption. However the GPU is designed to launch small amounts of data in parallel with only several instructions, in other words the GPU swap, switch threads very fast, they are extremely lightweight. In a typical system, thousands of threads are waiting to work. While the CPU only run up-to 24 threads on a hex-core processor. They can execute a single operation on comparatively large set of data with only one instruction. Although this can be extremely cost-wise operation on the GPU.

\section{GPUs as computing units}

A insight of the architecture of GPU can give a idea of  why it outperforms the CPU on various benchmarking.

The GPU, unlike its CPU cousin, has thousands for registers per SM (streaming multiprocessor), this are  arithmetic processing units. An SM can thought of like a multi-thread CPU core. On a typical CPU has two, four, six or eight cores. On a GPU as many as N SM core. We can see this in the figure \ref{fig:gpu_cpu}. For a particular calculation, all the stream
processors within a group execute exactly the same instruction on a particular data stream, then the data is sent to the upper level, the host (CPU). \cite{cook}

As commonly named CUDA cores are the number of processors in a single NVIDIA GPU chip. For example one of the first GPU capable of running CUDA code was the NVIDIA 9800 GT, which had 112 cores, while the latest high-end GPU GTX 980 has 2048 cores.

\begin{figure}[htbp]
	\centering
		\includegraphics[width=0.42\textwidth]{Figures/GPU_CPU.png}
		\rule{35em}{0.5pt}
	\caption[Architecture of a GPU]{Architecture of a NVIDIA GeForce GTX 580}
	\label{fig:gpu_cpu}
\end{figure}


Each CUDA core  can execute a sequential thread, just like a CPU thread, which NVIDIA calls it Single Instruction, Multiple Thread (SIMT). In addition all cores in the same group execute the same instruction at the same time, much like classical SIMD (Single instruction, multiple data) processors. SIMT handles conditionals somewhat differently than SIMD, though the effect is much the same, where some cores are disabled for conditional operations, in other word a single instruction is executed throughout the device.

Being able to efficiently use a GPU for an application requires to expose the inherent data-parallelism Optimized for low-latency, serial computation. This can be seen in contrast with a CPU, which is optimized for sequential code performance, fast switching registers  and sophisticated control logic allowing to run single complex programs as fast as possible, which is not possible on the GPU. Memory management is very important for GPUs. this refers how to allocate memory space and transfer data between host (CPU) and device (GPU). While the CPU memory hierarchy is almost non-existent, on the GPU inherent data is important. In figure \ref{fig:arch} different levels of memory can be observer between the host and the device, which differs form the CPU \cite{hwu}.

\begin{figure}[htbp]
	\centering
		\includegraphics[width=0.68\textwidth]{Figures/arch.png}
		\rule{35em}{0.5pt}
	\caption[Host and Device]{Memory transfer between the host and device}
	\label{fig:arch}
\end{figure}


On the GPU precision and optimization are very important but there is a penalty for choosing performance or precession. All the GPUs are optimized for single precision floating operations, 24 bit size, Also provides double precision point, size of  53 bits. This is using the standard notation IEEE 754. Normally the GPU uses single precession(SP) by default, if choosed double precision (DP), normally there is a penalty of  2x - 4x seed-up. \cite{precision}
Libraries such as CUBLAS and CUFFT provides useful information how NVIDIA handles floating point operations under the hood.


\section{Programming on GPUs}

There exist, among many, two main computing platforms, NVIDIA's Compute Unified Device Architecture (CUDA), and Khronos's Open Computing Language (OpenCL). NVIDIA's CUDA provides the necessary tools, frameworks and library to programs parallel computing, but for there GPUs. While OpenCL is a open standard framework meaning that is possible to do parallel computing on other GPUs, like on AMD cards. Programmers can easily port their code to others graphics cars.  However CUDA has more robust debugging and profiling for GPGPU computing. This two frameworks are developed to be close to the hardware layer, using C programming language. CUDA provides both a low level API and a higher level API. Those who are familiar to OpenCL and CUDa, can easily modify their code to work on either platform.\cite{hwu}


The CUDA programming model views the GPU as an accelerator processor which calls parallel programs  throughout all the SMI. This programs are only called on the device and are called kernels, which launch a large amounts of threads to execute CUDA code. The basic idea of programming on a GPU is simple. We can observer this in the figure \ref{fig:cycle}

\begin{figure}[htbp]
	\centering
		\includegraphics[width=0.3\textwidth]{Figures/cycle.png}
		\rule{35em}{0.5pt}
	\caption[Programming Cycle]{Programming Cycle between the CPU and GPU}
	\label{fig:cycle}
\end{figure}


\begin{itemize}
\item Create memory(data) for the host (CPU) and devices (GPUs)
\item Send the data host memory to the highly parallel device.
\item Do something with data on the device, e.g. matrix multiplication,calculation, parallel algorithm.
\item Return the data from the device to the host.
\end{itemize}

The structure of CUDA reflects the coexistence of CPU and GPUs. The CUDA code is a mixture of both host code and device code. The CUDA C compiler is called NVCC. The host code is the standard low level ANSI C language. The device code is marked is CUDA keywords for identifying data-parallel functions and has a extension file .cu.

When a kernel is launched, executed by a large amount of threads, where they are organized as a one, two or three dimensional grid of thread blocks. A thread is the simplest executing process. It consists of the code of the program, the particular point where the code is being executed. \cite{hwu}. Many threads form a block, and many blocks form a grid. CUDA handles the execution of the random-access threads, which take up-to very few clock cycles in comparison to CPU threads. The threads per block can be observer in figure \ref{fig:grid}. All the threads in a kernel can access the global memory, figure \ref{fig:arch}.

Each of the threads can be access by implicit variable that identifies its position within the thread block and its grid. In a case of 1-D block. \cite{example}

$$blockIdx.x \times blockDim.x + threadId.x$$

\begin{figure}[htbp]
	\centering
		\includegraphics[width=0.6\textwidth]{Figures/grid.png}
		\rule{35em}{0.5pt}
	\caption[Part of the CUDA's 2D grid]{Part of a 2D CUDA's thread grid, divided in blocks, each block with it’s own respective threads.}
	\label{fig:grid}
\end{figure}

In CUDA. host and device have separate memory spaces. This can bee seen on the host and device with the DRAM(Dynamic random-access memory) data. For example a NVIDIA GTX 660m comes with 2GB of memory, which is the global memory for the device. As told the host and device allocates data. The programmer needs to send data from the host memory to the device's global memory. We can see this in the figure  \ref{memorySpace:arch}.  Once the memory is transfer back to the host, is completely necessary to free the memory from the device and host. This is typically done with free or delete on C/C++. The CUDA's Application Programming Interface (API) functions performs this activities on behalf of the programmer.   \cite{hwu}

\begin{figure}[htbp]
	\centering
		\includegraphics[width=0.6\textwidth]{Figures/memorySpace.png}
		\rule{35em}{0.5pt}
	\caption[Memory Space GPU and CPU]{Separate memory spaces for the CPU and GPU}
	\label{fig:memorySpace}
\end{figure}

\subsection{Vector Addition Example}

A simple example of a vector addition to show the comparison between the GPU and CPU, input, two list of number which is sum up each corresponded element to produce a final output with the addition of both list. Figure \ref{fig:sum} shows this process. \cite{example}

\begin{figure}[htbp]
	\centering
		\includegraphics[width=0.45\textwidth]{Figures/sum.png}
		\rule{35em}{0.3pt}
	\caption[Vector Addition Example]{Simple Vector Addition Example}
	\label{fig:sum}
\end{figure}

\subsubsection{CPU Code}

This first example illustrates the CPU code executed in a single thread. The code is straight forward to understand. First create the memory for each array, A, B and C with size N. Then calculate the sum of the two vectors with the function \textit{add}. As we can see in the function \textit{add}, we use the while loop to go through each element of the arrays A and B, which are added into a single array C.

\begin{lstlisting}[language=C++, caption={CPU Vector Addition}]
#include <iostream>

#define N 100

void add( int *a, int *b, int *c );

int main()
{
    int A[N], B[N], C[N];
    
    //fill the arrays with values
    for(int i = 0; i < N; i++){
        A[i] = 1;
        B[i] = i;
    }
    
    add(A, B, C);
    
    //Display the results
    for (int i = 0; i < N; i++) {
        std::cout << A[i] << ", " << B[i] << ", " << C[i] << std::endl;
    }
    
    return 0;
}

void add( int * A, int * B, int * C )
{
    int index = 0;
    
    //go through each index of the arrays and make the operation
    while(index < N){
        C[index] = A[index] + B[index];
        index++;
    }
}

\end{lstlisting}

We can notice if we set N to be a large number, the function \textit{add} could take a large amount of time to execute. But the example only illustrates the used of the CPU as a single core, however nowadays CPUs commonly have around 4-8 cores. To be able to execute the previous code on all the cores available in the CPU, threads are needed to be implemented. But you would need reasonable amount of code and debugging to make that happen. Also is a complicated task to schedule all the threads in the CPU. 

\subsubsection{GPU Code}

We can accomplish the same operation very similar in the GPU with CUDA. First create CPU and GPU memory with there corresponded code. Send the CPU memory to the device, make calculations on the highly parallel GPU, finally return the results the CPU.

\begin{lstlisting}[language=C++, caption={GPU Vector Addition}]
#include <iostream>

#define N 100

// CUDA KERNEL
__global__ void add( int *a, int *b, int *c );

int main()
{
    int a[N], b[N], c[N];
    int *dev_a, *dev_b, *dev_c;

    // allocate the memory on the GPU
    cudaMalloc( (void**)&dev_a, N * sizeof(int) ) );
    cudaMalloc( (void**)&dev_b, N * sizeof(int) ) );
    cudaMalloc( (void**)&dev_c, N * sizeof(int) ) );
    
    //allocate the memory on the CPU
    for(int i = 0; i < N; i++){
        A[i] = 1;
        B[i] = i;
    }
    
    //calculate the vector addition in the GPU
    add<<<N,1>>>( dev_a, dev_b, dev_c );
    
    //copy back the result from the GPU to the CPU 
    cudaMemcpy( c, dev_c, N * sizeof(int), cudaMemcpyDeviceToHost ) );
     
     
    //Display the results
    for (int i = 0; i < N; i++) {
        std::cout << A[i] << ", " << B[i] << ", " << C[i] << std::endl;
    }
    
    cudaFree( dev_a );
    cudaFree( dev_b );
    cudaFree( dev_c );
    
    return 0;
}
\end{lstlisting}

As we can see the function \textit{cudaMalloc} and \textit{cudaFree} are very similar to the C code functions \textit{malloc} and \textit{fee} for allocating memory and deleting memory.

CUDA automatically spams the threads to it correspondent block, so we only need to access the index of the block and pass it to the index arrays. To parallel code will stop in-till the block index reaches the number of elements of the arrays, N.

\begin{lstlisting}[language=C++, caption={GPU Vector Addition}]
void add( int * A, int * B, int * C )
{
    // handle the data at this index if (tid < N)
    int index = blockIdx.x; 
    if(index < N)
        c[index] = a[index] + b[index];
}
\end{lstlisting}

The biggest difference between the CPU code and the GPU code is how threads are managed with-in the process. The CPU code has only one thread, thrust one loop, however if we want to expand the CPU code to multiples threads, extra loops are required for each additional thread. Based on this idea, the GPU code launches to code throughout every threads accessible by the device chip. 

\vspace{3.2em}

Finally, this chapter provided a quick overview of heterogeneous programming in a modern context. CUDA  enhance the C language with parallel computing support. Which is possible to launch  enormous amounts of parallel threads, oppose of few threads on the CPU. The number of GPU cores will continue to increase in proportion to increase in available transistors as silicon process improve. In addition, GPUs will continue to go through vigorous architectural evolution. Despite their demonstration high performance on data-parallel applications. \cite{hwu}









% Chapter Template

\chapter{Introduction to Domain Wall Dynamics under Nonlocal STT} % Main chapter title

\label{Introduction to Domain Wall Dynamics under Nonlocal STT} % Change X to a consecutive number; for referencing this chapter elsewhere, use \ref{ChapterX}

\lhead{Chapter 2. \emph{Introduction to Domain Wall Dynamics under Nonlocal STT}}% Change X to a consecutive number; this is for the header on each page - perhaps a shortened title


This chapter is a brief overview of the theory of spintronics and the study of Domain Wall Dynamics under Nonlocal Spin-Transfer-Torque. Which quantitatively test the effects of spin-diffusion, on real Domain Wall (DW) structures, by numerically implementing the Zhang-Li model on a NiFe soft nanostrip. The numerical method used for the solution is a the method known as Finite Differences in the Time Domain (FDTD) on a 3d cell grid with whose integration is done using a 4th order Runge-Kutta integration (RK4).

\section{Theory}

 The electrons not only carry an elementary
unit of charge $e$, but also carries an elementary unit of angular momentum. Whenever we produce an electrical current by inducing
motions of electrons, it could indeed be viewed as a collection of little magnets that are moving around (see Figure \ref{fig:electron}). In other words, any electron charge transport is simultaneously accompanied by a transport of spin, or magnetic moment carried by these electrons \cite{cornell}.

\subsection{Spintronics}

Spintronics is a new type of electronics that exploits the spin degree of freedom of an electron in addition to its charge \cite{spinz},  figure \ref{fig:electron}. The interest is motivated by the quest to understand basic physical principles underlying the electron and spin interactions in materials and possible technological applications. The field of spintronics has attracted massive interest since the discovery of giant magnectroresistence (GMR) effect in 1988 by Albert Fert and Peter Gr\"{u}nberg who were awarded the 2007 Nobel Prize in physics. The GMR effect has been widely used in hard disk drives (HDD), which have deliver a huge impact on industries and consumer electronics. Spintronics is a promising technology which will complement the present electronics with addition "spin" quantum freedom to charge freedom that is currently used in devices \cite{nonlocalspin}.

\subsection{Spin Transfer Torque }

A torque is simply a time rate of change of angular momentum \cite{spintransfer}. Hence, spin transfer torque (SST) occurs when spins flowing from one layer to another can reorient the magnetization in the layers, see figure \ref{fig:DWspin}. The magnetization of the ferromagnet changes the flow of spin angular momentum by exerting a torque on the flowing spins to reorient them, and therefore the flowing electrons must exert an equal and opposite torque on the ferromagnet. This torque that is applied by non-equilibrium conduction electrons onto a ferromagnet is what we will call the spin transfer torque \cite{spintransfer}.

Spin current which is a flow of spin angular momentum, is generated in addition to the charge current. The spin current normally appears in ferromagnets. However, it should be able to be generated in non-magnets. The simplest method of generating a spin-polarized current in a metal is to pass the current throughout a ferromagnetic material. A common application is the GMR as mentioned before \cite{handbookspin}.

\begin{figure}[htbp]
	\centering
		\includegraphics[width=0.42\textwidth]{Figures/electron.png}
		\smallskip
	\caption[Electron carries spin, charge and magnetic]{Electrons not only carries charge, but also spin and magnetic properties \cite{spinimg}. }
	\label{fig:electron}
\end{figure}


Spin polarized transport occurs naturally in any materials which have a spin imbalance between spin-up and spin-down at the Fermi Level. It occurs at spin-down electrons is nearly identical, but states are shifted in energy with respect to each other. The Fermi level is the highest energy level which an electron can occupy at the absolute zero temperature. Since at abosulte zeo temperature the electrons are all in the lowest energy state hence the Fermi level is in between the valence band and the conduction band \cite{handbookspin}.

\subsection{Domain Wall}

An abrupt in magnetization at the boundary of two anti-aligned domains is not a favorable condition. Domain walls form between such domains as means of minimizing the energy of the two anti-aligned domains. Domains walls are transitions layers in which the magnetization changes gradually from on magnetization to another. In other words the boundaries between regions of uniform magnetization. The gradual change prevents the large increase in exchange energy that would accompany an abrupt change in the magnetization angle. Common domain wall geometric include Bloch walls, N$\acute{e}$el walls and vortex walls \cite{spindomain}. In this study only two DW are analyzed the Vortex Wall and the Asymmetric Transverse Wall.

\begin{description}
  \item[Vortex Wall (VW)] \hfill \\
   In the case of Vortex wall the magnetization rotates in the plane perpendicular to the domain wall, but the local magnetization is wrapped around a single vortex point, see figure \ref{fig:dw}.
   
 \item[Asymmetric Transverse Wall (ATW)] \hfill \\
 The transverse wall has a reflection symmetry about a line perpendicular to the strip axis, and a lack of symmetry about the center line of the strip. However, asymmetric transverse wall, is the absence of that symmetry, see figure \ref{fig:dw}.
 
\end{description}

\begin{figure}[htbp]
	\centering
		\includegraphics[width=0.55\textwidth]{Figures/dw.png}
		\smallskip
	\caption[Domain Wall VW, ATW]{Vortex Wall (VW) and Asymmetric Transverse Wall (ATW) \cite{claudio}.}
	\label{fig:dw}
\end{figure}

\subsection{Spin Torque in Domain Walls}

Domain walls are the basis for various spintronics devices that uses magnetic momentums, in other words spin of electronics, the used of the spin degree of freedom. The figure \ref{fig:DWspin} illustrates a micromagnetic model of the domain wall trapped in a nanowire. The domain wall can be pushed along the wire in a controllable manner by applying an external magnetic field or by passing an electrical current through the wire \cite{dwwire}.

\begin{figure}[htbp]
	\centering
		\includegraphics[width=0.6\textwidth]{Figures/DWspin.png}
		\rule{35em}{0.52pt}
	\caption[Domain Wall nanowire]{Domain Wall in a nanowire while passing a current}
	\label{fig:DWspin}
\end{figure}

The energy of the incoming  carrier is no the only factor that determines whether or not it passes to the other side of domain wall, the spin also must be taken into account. Since each spin orientation experiences a different potential. Simulation of such properties is necessary.

Spin Torque induced domain wall motion opens up a host of possibilities for applications. The success of spintronics untimely depends on out ability to precisely  control the polarization of electrons transported within the actual thin film structure \cite{ferro}. Advances in spintronics recognized by 2007 Nobel Prize in Physics have enable over the last decade advances in computer memory, in hard drives, this is a metal based structures which utilize magnetoresistive effects to save and read data from a magnetic disk \cite{handbookspin}. An interesting application using this idea is new design for a different type memory disk drive called racetrack memory by Parkin in 2008\cite{racetrack}. The racetrack memory stores bits along a single ferromagnetic wire. To write and read information, a current is applied along the wire that moves the bits to writing or reading unit.
 
\section{Domain Wall Dynamics under Nonlocal STT}

The motion of domain walls due to spin transfer torque (STT) of electrons has been studied theoretically and experimentally. Furthermore, moving magnetic domain walls using electric currents via spin-torque effects, is one the recent developing in spintronics. We analyze a moving domain wall on a soft nanostrip, because it concentrates all of the magnetization non-uniformity, which acts as a built-in detector for spin torques. The inclusion of STT into micromagnetics has up to now been performed with local terms that express the STT as a function of only the local magnetization \cite{claudio}.
 
\subsection{Theoretical Approaches}

The inclusion of STT into micromagnetics has up to now been performed with local terms that express the STT as a function of the local magnetization only. The magnetization dynamics is described by the classical Landau-Lifshitz-Gilbert (LLG) equation \cite{claudio}, expanded with a STT variable \ref{eq:llg}. 

\begin{equation}  \label{eq:llg}
	\frac{\partial \vec{m}}{\partial t} = \gamma_0\vec{H}_{eff} \times \vec{m} + \alpha \vec{m} \times \frac{\partial \vec{m}}{\partial t} - \vec{T}
\end{equation}

This novel idea of incorporating spin torque into the LLG equation has itself been incorporated into a model proposed by Zhang-Li in 2004 \cite{zhang2004}. The LLG equation \ref{eq:llg} is incorporated effects of a spin-polarized current in a magnetic system, and the resulting spin transfer. They develop a form for the spin torque based on the spatial variation of the magnetization, as especially appropriate approach for domain walls. Then in 2005 the same authors Zhang-Li extended this idea working out the difference between the adiabatic and non-adiabatic torque contributions. Which lead to an even longer magnetization dynamics equation \cite{zhang} \cite{spindomain}.

\begin{equation} \label{eq:zhang}
 \frac{\partial \delta \vec{m} }{\partial t} =  D_{0}\bigtriangledown^{2} \delta \vec{m} - \frac{1}{\tau_{sd}} \delta \vec{m} \times \vec{M} - \frac{1}{\tau_{sf}}\delta \vec{m} +(\vec{\mu} \cdot\vec{\bigtriangledown} )\vec{M}
\end{equation}

The equation \ref{eq:zhang} is referred to Zhang-Li model, represents a non-adiabatic spin torque, with the presence of spin diffusion. Spin diffusion is a process by which magnetization is exchanged spontaneously between spin, which spin is able to accumulate in metals. The associated diffusion current flows in all directions, giving rise to nonlocal effects. The diffusion term of the equation \ref{eq:zhang} which carriers’  drift-diffusion equation implies that the spin density does not depend solely on the local magnetization, which gives rise of nonlocal magnetics effects \cite{claudio}. 

Amongst the rapidly growing variety of proposed and developed spin structures, nonlocal spin detection devices, where measurement and current excitation paths are spatially separated, have recently gained a prominent position \cite{spinz}.

\subsection{Experiment}

We Quantitatively test the effects of spin diffusion, on real Domain walls structures, this is done by numerically solve the Zhang-Li model into micro-magnetics, using the equation \ref{eq:zhang}. Zhang-Li research  \cite{zhang} initially solves analytically the diffusion equation \ref{eq:zhang}, However, ignoring the term of spin diffusion. In this numerically simulation we solve such equation using the spin diffusion term.

The sample considered is a 300 nm wide and 5 nm tick NiFe soft nanostrip. This dimensions are widely used for experimental use. Two Domain walls are used a Asymmetric Transverse Wall (ATW) and a Vortex Wall (VW). ATW maps of magnetization components of non equilibrium spin accumulation under a uniform current density with $D = 0, 1$ and $10 nm^2 / ps$. See figure \ref{fig:atw}.

\begin{figure}[htbp]
	\centering
		\includegraphics[width=0.64\textwidth]{Figures/ATW.png}
		\smallskip
	\caption[Asymmetric Transverse Wall results]{Asymmetric Transverse Wall (ATW) results \cite{claudio}.}
	\label{fig:atw}
\end{figure}

Vortex Wall (VW) same as for ATW, we point out the noticeable effect of the diffusion constant around the vortex core, which is the smallest feature of the wall. See figure \ref{fig:vw}.

\begin{figure}[htbp]
	\centering
		\includegraphics[width=0.64\textwidth]{Figures/VW.png}
		\smallskip
	\caption[Vortex Wall results]{Vortex Wall results \cite{claudio}. }
	\label{fig:vw}
\end{figure}

\section{Numerical Solution}

The equation \ref{eq:zhang} is physically realistic, However, computationally expensive. Therefore we numerical methods to solve such equation. The numerical methods used for the solution is a the method known as Finite Differences in the Time Domain (FDTD) whose integration is done using a 4th order Runge-Kutta integration.

\subsection{Finite differences in the time domain}

The finite difference in the time domain (FDTD) method  is able to solve complicated problems. However, it is generally computationally expensive. Solutions may require a large amount of memory and computation time \cite{ufdtd}. FDTD is a numerical analysis technique use for approximating solutions to the associates system of differential equations. The method belongs in the general class of grid-based differential numerical modeling methods. Please read reference \cite{methods} for more information about the demonstration of such numerical methods for this section.

The FDTD method essentially uses a weighted summation of functions values at neighboring points to approximate the derivate at a particular point, in this case a point in a 3d grid. The result for each cell is based on the results from the cell and its neighbors at the previous time-frame, figure \ref{fig:fdtd}.  

\begin{figure}[htbp]
	\centering
		\includegraphics[width=0.78\textwidth]{Figures/fdtd.png}
		\smallskip
	\caption[FDTD grid]{The result for each cell is based on evaluating the derivate cell neighbors \cite{methods}.}
	\label{fig:fdtd}
\end{figure}

The magnetization is sampled on a uniform rectangle mesh at points $(x_0 + i\bigtriangledown_x, y_0 + j\bigtriangledown_y, z_0 + k\bigtriangledown_z)$. The computational cell is centered about the sample point with dimensions. $\bigtriangledown_x \times \bigtriangledown_y \times \bigtriangledown_z$ \cite{methods}.

Looking at the equation \ref{eq:zhang}, we need a method to calculate the first and second derivate. With the Taylor expansion we are able to perform such calculation. The Second order Taylor expansion readily yields expressions for the first and seconds central derivates. First and second-order derivates of the magnetization components in order to define the divergence  of the magnetization $(\nabla \cdot m)$, and the components of the exchange field $(\nabla^2m)$, respectively. The magnetization components along boundaries also need to be evaluated in order to define surface charges $(m \cdot n)$. Boundary conditions need to be incorporated in the evaluated of the effective field without loss of accuracy. 

Consider a regular, differentiable one-dimension  scalar function $f(x)$ sampled at regular intervals, a, see figure  \ref{fig:bound}. Second order Taylor expansion readily tiles expressions for the first and seconds central derivates that are widely used in numerics, namely $\frac{df}{dx} = \frac{f_{i+1} - f_{i-1}}{2a}$ and $\frac{d^2f}{dx^2} = \frac{f_{i+1} - 2f_i + f_{i-1}}{a^2}$ \cite{methods}.

\begin{figure}[htbp]
	\centering
		\includegraphics[width=0.65\textwidth]{Figures/bound.png}
		\smallskip
	\caption[Sampled at regular intervals a, Taylor expansion]{Sampled at regular intervals a, (a) Function of inside the grid. (b) Mesh points second to closest to boundary. (c) Mesh points closet to boundary}
	\label{fig:bound}
\end{figure}

However, the numerical derivation of the structure of a simple Bloch wall using such expressions soon reveals that second order Taylor expansion ledes to restricted accuracy. Fourth order expansion as actually been found to prove much superior \cite{methods}.

Taylor expansion of the function $f(x)$ around $x=x_i$ yields where $f^{(k)}(x_i) = f(x)$ if $k=0$

$$f(x) = \sum\limits_{k=0}^{\infty} \dfrac{(x-x_i)^k}{k!}f^{(k)}(x_i) = \sum\limits_{k=0}^{\infty} \dfrac{(x-x_i)^k}{k!}f^{(k)}$$

Applying the previous equation to nearest and next nearest neighbor to grid pint i and truncation the the 4th order yields a set of four equations

The set of linear equations provide numerical estimates for the first, second, third and fourth derivatives of $f$ at any given point $i$. The general form of the first and second derivate based on second nearest neighbors expansion reads:

\begin{align} \label{eq:nn}
f^{(1)}_i &= \dfrac{f_{i-2}-8f_{i-1} + 8f_{i+1} - f_{i+2}}{12a} \\
f^{(2)}_i &= \dfrac{f_{i-2}+16f_{i-1} -30f_{i} + 16f_{i+1} - f_{i+1}}{12a^2}
\end{align}

The equation \ref{eq:nn} for the second derivate based on second nearest neighbors expansion solves for the laplacian operator in the Zhang -Li Model equation \ref{eq:zhang}. However, points close to the edges need to be evaluated for great precession. 

\subsubsection{Boundary conditions}

Expressions such as \ref{eq:nn} are valid when the grid point becomes closet or next-to-closet to the boundary of the magnetic box. Specific accuracy preserving, expansion need to be worked out. The general principal in the present approach is to replace equations that are missing because of the lack of grid points outside the magnetic volume by equations including explicit reference to boundary conditions \cite{methods}.

Consider first a point second to closet to bound, \ref{fig:bound}-b. Grid point $i + 1$ is missing for this particular geometry. However, defining $x_R$ as the right boundary coordinate along the $x$ axis. The $f^{(1)}(x_R)$ to be know along the boundary to be replace by the derivate of Taylor's expansion \cite{methods}.

\begin{equation} \label{eq:taylor}
f^{(1)}(x) = \sum\limits_{k=0}^{\infty} \dfrac{(x-x_i)^{k-1}}{(k - 1)!}f^{(k)}(x_i)
\end{equation}

Using \ref{fig:bound}-b. $x_R - x_i = 3a/2$ becomes \cite{methods}.

\begin{align}
\begin{bmatrix} \label{eq:matrix4}
    -2a & \dfrac{(-2a)^2}{2!} & \dfrac{-(2a)^3}{3!} & \dfrac{(-2a)^4}{4!}\\
    -a & \dfrac{(-a)^2}{2!} & \dfrac{(-a)^3}{3!} & \dfrac{(-a)^4}{4!}\\
    a & \dfrac{(a)^2}{2!} & \dfrac{(a)^3}{3!} & \dfrac{(a)^4}{4!}\\
    2a & \dfrac{(2a)^2}{2!} & \dfrac{(2a)^3}{3!} & \dfrac{(2a)^4}{4!}
\end{bmatrix}
\begin{bmatrix}
    f_i^{(1)}  \vphantom{ \dfrac{d^4}{d} }\\
    f_i^{(2)}  \vphantom{ \dfrac{d^4}{d} } \\
    f_i^{(3)}  \vphantom{ \dfrac{d^4}{d} } \\
    f_i^{(4)}  \vphantom{ \dfrac{d^4}{d} }
\end{bmatrix}
=
\begin{bmatrix}
    f_{i-2} - f_{i}    \vphantom{ \dfrac{d^4}{d}} \\
    f_{i-1} - f_{i}    \vphantom{ \dfrac{d^4}{d}} \\
    f_{i+1} - f_{i}    \vphantom{ \dfrac{d^4}{d}} \\
    f^{(1)}(x_R)   \vphantom{ \dfrac{d^4}{d}}
\end{bmatrix}
\end{align}

Similarly, for a point closet to boundary, reference \ref{fig:bound}-c, grid points i + 1 and i + 2 are missing. The two first equation of ... need now to be replaced by a single equation, whilst the two remaining equations need to be truncated to the third order. For the geometry illustrated in \ref{fig:bound}-c, the minimal set of equations now reads \cite{methods}.

\begin{align} \label{eq:matrix3}
\begin{bmatrix}
    -2a & \dfrac{(-2a)^2}{2!} & \dfrac{-(2a)^3}{3!}\\
    -a & \dfrac{(-a)^2}{2!} & \dfrac{(-a)^3}{3!}\\
    1 & \dfrac{(+a)}{2} & \dfrac{(+a/2)^3}{2!}
\end{bmatrix}
\begin{bmatrix}
    f_i^{(1)}  \vphantom{ \dfrac{d^4}{d} }\\
    f_i^{(2)}  \vphantom{ \dfrac{d^4}{d} } \\
    f_i^{(3)}  \vphantom{ \dfrac{d^4}{d} }
\end{bmatrix}
=
\begin{bmatrix}
    f_{i-2} - f_{i}    \vphantom{ \dfrac{d^4}{d}} \\
    f_{i-1} - f_{i}    \vphantom{ \dfrac{d^4}{d}} \\
    f^{(1)}(x_R)   \vphantom{ \dfrac{d^4}{d}}
\end{bmatrix}
\end{align}

In both cases, and second derivatives and fully determined provided $f^{(1)}(x_R)$ be known along the boundary. For further reference please read\cite{methods}. For implementation of the laplacian boundaries conditions please read Chapter \ref{Implementation of Domain Wall Dynamics under Nonlocal STT}.

The main advantages of the finite difference approach is easy to implement, simplicity of meshing, efficient evaluation of the magnetization energy, and the accessibility of higher order methods. The main disadvantage of this approach is the sampling curved boundaries with a rectangular mesh, resulting in some what discrete approximation. In addition, it could produce a significant error in the evaluation.

\subsection{Fourth order Runge and Kutta method}

 Modern numerical algorithms for the solution of ordinary differential equations are based on the method of the Taylor series. Algorithm such as the Runge-Kutta method are constructed so they give an expression depending of the parameter $(h)$, in other works the step as an approximate solution of the first terms of the Taylor series. The method is able to accurately solve a wide range of problems, but it is generally computationally expensive. Solutions require large amount of memory and computational time \cite{numerical}.

There exist several other computational numeric methods to solver such equations, methods such as the Euler integrator, the Midpoint Method and the Runge-Kutta fourth order (RK4) integrator method can solve differential equations. However, they differ in the numerically approximation and computation time. The RK4 is used for this simulation because its numerically more accurate when compared to the others methods.

The RK4 method differs widely from the Euler method and the Midpoint method. The Euler method is the simplest, the derivative at the starting point of each interval is extrapolated to find the next function value, see figure \ref{fig:euler}. Euler method only has first order accuracy while the RK4 its fourth order integrator \cite{numerical}.

\begin{figure}[htbp]
	\centering
		\includegraphics[width=0.6\textwidth]{Figures/euler.png}
		\smallskip
	\caption[Euler Method]{Euler Method, Is the simplest approximate to solver differential equation or numerically solve equations.}
	\label{fig:euler}
\end{figure}

RK4 goes as follows:

\begin{equation} \label{eq:rk4}
y_{n+1} = y_{n} + 1/6 K_{1} + 1/3 K_{2} +1/3 K_{3} + 1/6 K_{4}
\end{equation}
where
\begin{equation}
\begin{split} \label{eq:rksplit}
K_{1} &= h \dot f(x_{n}, y_{n}) \\
K_{2} &= h \dot f(x_{n} + h/2, y_{n} + k_{1}/2) \\
K_{3} &= h \dot f(x_{n} + h/2, y_{n} + k_{2}/2) \\
K_{4} &= h \dot f(x_{n} + h, y_{n} + k_{3})
\end{split}
\end{equation}

As the equations shows, each step, the derivative is evaluated four times, once at the initial point, twice at trial midpoints, and once at a trial endpoint. From these four values, the final value is calculated, just like the equation \ref{eq:rk4}.

\begin{figure}[htbp]
	\centering
		\includegraphics[width=0.5\textwidth]{Figures/rk4.png}
		\smallskip
	\caption[Fourth order Runge and Kutta Method]{Fourth order Runge and Kutta method, each step the derivative is evaluated four times. }
	\label{fig:kutta}
\end{figure}


\vspace{4.0em}

In conclusion, the simultaneous solution of the diffusive Zhang and Li model \ref{eq:zhang} has uncovered a qualitatively new feature of the spin-transfer torque effect in the presence of spin diffusion. Namely the dependence of the steady-state DW velocity on DW structure \cite{claudio}. In summarize, we quantitatively test the effects of spin diffusion, on real Domain walls structures for ATW and VW. This is done by numerically solve the Zhang-Li model into micro-magnetics. The numerical methods used to solve such model as mentioned is the FDTD on a 3D cell grid with whose integration is done using RK4.



%Ind addition, these results offers a starting point to study multilayer structures like spin-value nano strips, where the understanding of the observed increased efficecnctiy of SST to drive DW's still remains elusive. 




% Chapter Template

\chapter{Introduction to Domain Wall Dynamics and a Implementation with CUDA} % Main chapter title

\label{Introduction to DW Dynamics} % Change X to a consecutive number; for referencing this chapter elsewhere, use \ref{ChapterX}

\lhead{Chapter 3. \emph{Introduction to DW Dynamics}}% Change X to a consecutive number; this is for the header on each page - perhaps a shortened title

%----------------------------------------------------------------------------------------
%	SECTION 1
%----------------------------------------------------------------------------------------
This chapter is a overview of the theory and experiments behind Dr. Cluadio's work "Domain Wall Dynamics under Nonlocal Spin-Transfer Torque". This is a quantitatively test the effects of spin-diffusion, on real Domain Wall (DW) structures, by numerically implementing the Zhang-LI model into a NiFe soft nanostrip \cite{claudio}. The implementation takes advantage of the highly parallel process capabilities of the GPU.


\section{Theory}

Spintronics is a new type of electronics that exploits the spin degree of freedom of an electron in addition to its charge. It is expected that electronics technology and devices will be faster, compacter and more energy-saving \cite{spinz}

\cite{ferro}

An abrupt in magnetization at the boundary of two anti-aligned domains is not a favorable condition. Domain walls form between suc domains as means of minimizing the energy of the two anti-aligned domains. Domains walls are transitions layers in which the magnetization changes gradually from on magnetization to another.  The gradual change prevents the large increase in exchange energy that would accompany an abrupt change in the magnetization angle. \cite{spindomain}

\begin{figure}[htbp]
	\centering
		\includegraphics[width=0.5\textwidth]{Figures/vortex.png}
		\rule{35em}{0.5pt}
	\caption[Domain Wall - Vortex]{Domain Wall - Vortex}
	\label{fig:vortex}
\end{figure}

The study spin-diffuse effect within a continuously variable magnetization distribution, integrating with micromagenectis with diffuse model of Zhang and LI \cite{claudio}

Numerical Methods
\cite{methods}


Spin-transfer torque is a torque that exerts on a magnetization by conduction electron spins, in other words the angular momentum transferred from spins to magnetic  moment \cite{zhang}.

This has simulated research into domain wall (DW) dynamics, particularly those resulting from interactions with current passing through the DW via the phenomenon of spin momentum transfer (SMT) \cite{handbookspin}





Contrarily to charge, spin accumulate in metals, The associated diffusion current flows in all directions, giving rise to nonlocal effects, Beyond transport properties, conduction electrons spin resonance and spin pumping provide further testimonies for non-locality in spin transport. These works all refer  to samples consisting in piecewise uniform layers or blocks, magnetic or not. Of special significance to the present work in the non-collinear geometry where a spin current with polarization transverse to the magnetization exists, whose absorption in the vicinity of the surface of a magnetic layer creates a torque on the magnetization, known as spin transfer torque (SFF), 


%Within a nanostip wire 

We Quantitatively test the effects of spin diffusion, on real Domain walls structures, this is done by numerically solve the Zhang-Li model \cite{zhang} into micro-magnetics.
The Zhang Li model refers to:

which is the following equation.

Base on the work of Dr. Claudio \cite{claudio}

At first we investigate the steady-sate velocity regime of DWs in NiFe soft nanostrips. applying current densities similar to those reported in experiments. The results that we are going to obtain 

Experimentally measured spin-diffusion parameters are used, we want to the solution of. 

\begin{equation}
 \frac{\partial \delta \vec{m} }{\partial t} =  D\bigtriangleup \delta \vec{m} + \frac{1}{\tau_{sd}} \vec{m} \times \delta  \vec{m} - \frac{1}{\tau_{sf}}\delta \vec{m} - u \partial_{x}  \vec{m}
\end{equation}


The sample that is considerate is a 300 nm wide and 5 nm tick NiFe soft nanostrip. This dimensions are widely used for experimental use.

Therefore, a simultaneous solution of the diffusive Zhang and Li model together with the magnetization dynamics equation has uncovered a qualitatively new feature of the spin-transfer torque effect in the presence of spin diffusion.


Advances in spintronics recognized by 2007 Nobel Prize in Physics have enable over the last decade advances in computer memory, in hard drives, this is a metal based structures which utilize magnetoresisite effects to save and read data from a magnetic disk. \cite{handbookspin} 

Some application include racetrack technology by the IBM fellow scientific Parkin \cite{racetrack}

Base on this study numeric applications have been unfold. A interesting application using spintronics is new design for a new memory disk drive called racetrack memory by  Parkin in 2008\cite{racetrack}


%-----------------------------------
%	SUBSECTION 1
%-----------------------------------
\section{Domain Wall Dynamics on the GPU}


The implementation of the GPU of Dr. Claudio is based on launching several kernels into a single GPU node.


The differential evaluation in one-dimention,
The Second order Taylor expansion readily yields expressions for the first and seconds central derivates

$$ \dfrac{df}{dx} = \dfrac{f_{i+1} - f_{i-1} }{2a}$$

and

$$ \dfrac{d^{2}f}{dx^{2}} = \dfrac{f_{i+1} - 2f_{i}+f_{i-1} }{a^2}$$

\begin{figure}[htbp]
	\centering
		\includegraphics[width=0.4\textwidth]{Figures/taylor.png}
		\rule{35em}{0.2pt}
	\caption[Sampled at regular intervals a, Taylor expansion]{Sampled at regular intervals a}
	\label{fig:taylor}
\end{figure}

Taylor expasion of the function $f(x)$ around $x=x_i$ yields where $f^{(k)}(x_i) = f(x)$ if $k=0$

$$f(x) = \sum\limits_{k=0}^{\infty} \dfrac{(x-x_i)^k}{k!}f^{(k)}(x_i) = \sum\limits_{k=0}^{\infty} \dfrac{(x-x_i)^k}{k!}f^{(k)}$$

Applying the previous equation to nearest and next neares neigborar to grid pint i and tructation tht the 4th order yields a set of four equations:

\begin{align}
\begin{bmatrix}
    -2a & \dfrac{(-2a)^2}{2!} & \dfrac{-(2a)^3}{3!} & \dfrac{(-2a)^4}{4!}\\
    -a & \dfrac{(-a)^2}{2!} & \dfrac{(-a)^3}{3!} & \dfrac{(-a)^4}{4!}\\
    a & \dfrac{(a)^2}{2!} & \dfrac{(a)^3}{3!} & \dfrac{(a)^4}{4!}\\
    2a & \dfrac{(2a)^2}{2!} & \dfrac{(2a)^3}{3!} & \dfrac{(2a)^4}{4!}
\end{bmatrix}
\begin{bmatrix}
    f_i^{(1)}  \vphantom{ \dfrac{d^4}{d} }\\
    f_i^{(2)}  \vphantom{ \dfrac{d^4}{d} } \\
    f_i^{(3)}  \vphantom{ \dfrac{d^4}{d} } \\
    f_i^{(4)}  \vphantom{ \dfrac{d^4}{d} }
\end{bmatrix}
=
\begin{bmatrix}
    f_{i-2} - f_{i}    \vphantom{ \dfrac{d^4}{d}} \\
    f_{i-1} - f_{i}    \vphantom{ \dfrac{d^4}{d}} \\
    f_{i+1} - f_{i}    \vphantom{ \dfrac{d^4}{d}} \\
    f_{i+2} - f_{i}    \vphantom{ \dfrac{d^4}{d}} 
\end{bmatrix}
\end{align}

The set of linear equations provide numerical estimates for the first, second, third and fourth derivatives of $f$ at any given point $i$.

The general form of the first and second derivate based on second nearest neighbors expansion reads:

\begin{align*}
f^{(1)}_i &= \dfrac{f_{i-2}-8f_{i-1} + 8f_{i+1} - f_{i+2}}{12a} \\
f^{(2)}_i &= \dfrac{f_{i-2}+16f_{i-1} -30f_{i} + 16f_{i+1} - f_{i+1}}{12a^2}
\end{align*}


\begin{align}
\begin{bmatrix}
    -2a & \dfrac{(-2a)^2}{2!} & \dfrac{-(2a)^3}{3!} & \dfrac{(-2a)^4}{4!}\\
    -a & \dfrac{(-a)^2}{2!} & \dfrac{(-a)^3}{3!} & \dfrac{(-a)^4}{4!}\\
    a & \dfrac{(a)^2}{2!} & \dfrac{(a)^3}{3!} & \dfrac{(a)^4}{4!}\\
    1 & \dfrac{(3a)}{2} & \dfrac{(3a/2)^3}{3!} & \dfrac{(3a/2)^4}{4!}
\end{bmatrix}
\begin{bmatrix}
    f_i^{(1)}  \vphantom{ \dfrac{d^4}{d} }\\
    f_i^{(2)}  \vphantom{ \dfrac{d^4}{d} } \\
    f_i^{(3)}  \vphantom{ \dfrac{d^4}{d} } \\
    f_i^{(4)}  \vphantom{ \dfrac{d^4}{d} }
\end{bmatrix}
=
\begin{bmatrix}
    f_{i-2} - f_{i}    \vphantom{ \dfrac{d^4}{d}} \\
    f_{i-1} - f_{i}    \vphantom{ \dfrac{d^4}{d}} \\
    f_{i+1} - f_{i}    \vphantom{ \dfrac{d^4}{d}} \\
    f_{i+2}(x_R)   \vphantom{ \dfrac{d^4}{d}} 
\end{bmatrix}
\end{align}
     

\cite{methods}



Modern numerical algorithms for the solution of ordinary differential equations are also base on the method of the Taylor series. Each algorithm such as Runge-Kutta method are constructed so they give an expression depending of the parameter $(h)$, in other works the step as an approximate solution of the first terms of the Taylor series.

\subsection{Runge and Kutta}

There exist several computational numeric methods to solver such equations, methods like Euler, Midpoint Method and Runge-Kutta integrator method can solve this equations. The RG4 this method is used for the simulation because its numerically more accurate when compared to the others.

The RG4 method differs widely from the Euler method and the Midpoint method. The euler method is the simplest, the derivative at the starting point of each interval is extrapolated to find the next function value, see figure \ref{fig:euler}. The method is only has first order accuracy while RG4 its fourth order integrator.

\begin{figure}[htbp]
	\centering
		\includegraphics[width=0.7\textwidth]{Figures/euler.png}
		\rule{35em}{0.5pt}
	\caption[Euler Method]{Euler Method, Is the simplest approximate to solver differential equation or numerically solve equations.}
	\label{fig:euler}
\end{figure}

RK4 goes as follows:

\begin{equation} \label{eq:kg4}
y_{n+1} = y_{n} + 1/6 K_{1} + 1/3 K_{2} +1/3 K_{3} + 1/6 K_{4} 
\end{equation}
where

\begin{align*}
K_{1} &= h \dot f(x_{n}, y_{n}) \\
K_{2} &= h \dot f(x_{n} + h/2, y_{n} + k_{1}/2) \\
K_{3} &= h \dot f(x_{n} + h/2, y_{n} + k_{2}/2) \\
K_{4} &= h \dot f(x_{n} + h, y_{n} + k_{3})
\end{align*}

As the equations shows, each step the derivative is evaluated four times, once at the initial point, twice at trial midpoints, and once at a trial endpoint. From these four values, the final value is calculated, just like the equation is shown \ref{eq:kg4}

\begin{figure}[htbp]
	\centering
		\includegraphics[width=0.5\textwidth]{Figures/rk4.png}
		\rule{35em}{0.5pt}
	\caption[Fourth-order Runge and Kutta Method]{Fourth-order Runge and Kutta method, Each step the derivative is evaluated four times. }
	\label{fig:kutta}
\end{figure}

\cite{numerical}

%The basic structure is computational solve rungge and kutta of for other.

%-----------------------------------
%	SUBSECTION 2
%-----------------------------------


\subsection{Kernels}

The GPU implementation. the application reads

At initialize the applications first it allocates all the CUDA and c arrays.

To allocate a big chunk of memory in the Device 

\begin{lstlisting}[frame=none]
cudaMalloc
\end{lstlisting}
And C with

\begin{lstlisting}[frame=none]
malloc
\end{lstlisting}

In the initialization function it also reads the magenetizacion data from a especific file in this specific case from ``upVW-magn-2.5nm.data''


The initial values for the simulation are

\begin{table}[h]
\centering
\begin{tabular}{| l | l |}
\hline   
NX     & 480                                            \\
\hline   
NY     & 120                                            \\
\hline   
NZ     & 1                                              \\
\hline   
TX     & 1200.0                                         \\
\hline   
TY     & 300.0                                          \\
\hline   
TZ     & 5.0                                            \\
\hline   
u      & 1                                              \\
\hline   
D      & 1.0e\textasciicircum 3 nm\textasciicircum 2/ns \\
\hline   
tau sd & 1.0e\textasciicircum -3 ns                     \\
\hline   
tau sf & 25.0e-3 ns      \\
\hline   
\end{tabular}
\end{table}


\subsection{Threads}

The number of threads that are allocated within each kernel is a 2d grid.

\begin{table}[h]
\centering
\begin{tabular}{| l | l |}
\hline   
Threads per block  X   & 32       \\
\hline   
Threads per block Y     & 32         \\
\hline
\end{tabular}
\end{table}

Depending on the hardware configuration, each GPU can allocate different threads per block. To make a homogeneous grid space for each GPU a simple calculation is made.

\begin{lstlisting}[frame=none]
NXCUDA = (int)powf(2,ceilf(logf(NX)/logf(2)));
printf("NXCUDA = %i\n",NXCUDA);
NYCUDA = (int)powf(2,ceilf(logf(NY)/logf(2)));
printf("NYCUDA = %i\n",NYCUDA);
if((int)powf(2,ceilf(logf(NZ)/logf(2))) < 1)
	NZCUDA = 1;
else
        NZCUDA = (int)powf(2,ceilf(logf(NZ)/logf(2)));
//printf("NZCUDA = %i\n",NZCUDA);

//Setup optimum number of blocks
XBLOCKS_PERGRID = (int)ceil((float)NX/(float)XTHREADS_PERBLOCK); 
printf("XBLOCKS_PERGRID = %i\n",XBLOCKS_PERGRID);

YBLOCKS_PERGRID = (int)ceil((float)NY/(float)YTHREADS_PERBLOCK); 
\end{lstlisting}


The calculations are divided into two parts, the CPU code and GPU code. Most of the code is on the GPU. On the CPU only minor  process are taken place, like I/O to a .data.
On GPU is were all the computation is happening and simulation.

\subsection{CPU}
In the $initial_calculations$ functions it calculates the terms on magnetization components


the read magnetization data.
This function basically reads data from a .dat file and allocates the memory for each blocks, it reads

The file is divide into two blocks of data, the first block of 57600 rows are the coordinate X and the coordinate Y. Then the next 57600 rows by 3 columns are the magnetization data. Base on the information read the matrices of data is created.

here two data sets are created. The coordinates point data $(x,y)$ and the magnetization data $(x, y, z)$.

Afther this initilization data, the next step is to send this data, that is actually read on the CPU (host) to de GPU(device). 


First we print the Initial and final coordinates that read, this is to ensure that the values ared sucuefully.


\subsection{GPU}


Array are created on the on the Host and sento the Device using

In the type it can be either cudaMemcpyHostToDevice or cudaMemcpyDeviceToHost, depeding, if the memory thats is being copied is sent to the host or to the device.

\begin{lstlisting}[frame=none]
cudaMemcpy(dst, src, size_in_bytes, type);
\end{lstlisting}

after initialization the coordinates points an the magnetization data in the device are done, does values are sent to the GPU, with the function cudaMemcpy() and value set to cudaMemcpyHostToDevice.

In the Initialization of the calculations most of the arrays are filled up with values base on the data read from the .dat magenetization.

\begin{lstlisting}[frame=none]
__global__ void gsource(double *sm_out, double *matrix_in, double u, int grid_width);

__global__ void gm_x_source(double *tempx, double *tempy, double *tempz,
							 double *mx, double *my, double *mz,
							 double *sm_x, double *sm_y, double *sm_z,
							 int grid_width);
\end{lstlisting}


The function gsource

Makes the following calculation of the $double *matrix\_in$ or m


\begin{equation} \label{eq:gsource}
out[i] = (m[i-2] - 8.0*m[i-1] + 8.0*m[i+1] - m[i+2]) * \dfrac{u}{12 * deltaX }
\end{equation}

where

$$deltaX = \frac{TX}{NX}$$

This calculation is done for the arrays read from the .dat file, for dev\_mx, dev\_my and dev\_mz and are saved in a temporary arrays dev\_sm\_x, dev\_sm\_y, and dev\_sm\_z.

The method.

$gm\_x\_source$ calculates the coss producto of the array $m_{xyx}$ and $sm_{xyz}$, this is done twice.

THis data is saved on the arrays $dev_sm_{xyz}$,

Afther launching this two kernels the initial setup is done, the next step is the actual simulation using runge and kutta integrator.


\subsection{KG4}

As seed  in Runge and Kutta section, this method is implementation to numerically solve the differential equation. Intuitivelly the implementation on CUDA code is done with 4 kernls, where each kernel calculates respectivelly the order of the integrator. In the last term calculation is where all the magic occurr, the sum of the previous 3 calculated terms.

\begin{lstlisting}[frame=none]
__global__ void gterm1_RK1( . . .);
__global__ void gterm2_RK2( . . .);
__global__ void gterm3_RK3( . . .);
__global__ void gterm4_RK4( . . .);
\end{lstlisting}

Between each term calculation of RG4 laplacian calculation kernels are launched.

\begin{lstlisting}[frame=none]
__global__ void glaplacianx( . . . );
__global__ void glaplacianyboundaries( . . . );
__global__ void glaplaciany( . . . );
\end{lstlisting}

The final kernel is launched $void gterm4_RK4()$ obtain the array $deltam_{xyz}$, which is the final result of the RK4 integrator. This array is sent to the last step.

\subsection{effective values}

When the rg4 integretaor is done effective values are calculatated, this values sirve the porpese of calculation  the.


\begin{lstlisting}[frame=none]
__global__ void gm_x_sm( . . . );
__global__ void gu_eff( . . . );
__global__ void gu_eff_beta_eff( . . . );
__global__ void gbeta_eff( . . . );
__global__ void gbeta_diff( . . . );
\end{lstlisting}

The last kernel $ void gbeta\_diff( . . . );$ is where the two final arrays are obtain,
which then are sent to the CPU for the final calculation.

The final calculation is just the sum of all the elements of $beta_diff_num$ and $beta_diff_den$, there divided.
This final single values tells us...


This is the final step of the simulations this is where $beta\_diff$ is obtained. 

The final data is saved
\subsection{time}

When the simulation is done, it will repeat the process intil the values converges.


\section{Validation}

The validate the code, that is obtained from the simulation

Once obtain the results from the simulation, the results are saved into two seperated data sets. .eff and .spin. depending of the configuration of the application is possible to obtain the uVW or the. Because CUDA framework is highly parallel system is farly easy to obtain errenois data from the calculations, even setting up the threads per block incorrectly is possible to get data set that a wrong, or results that don't diverge. It is necessary that when finishing making changes to the code validating the results with a valid data set is done.

The validation is done by checking the output the simulation with a valid data set, the output of the validation application tells us the error factor of the current data with the valid set. So for each data set there is a threshold value, that can tell if the that is close enough to the results. A example of the validation performed.


\section{Help Kernels}


%----------------------------------------------------------------------------------------
%	SECTION 2
%----------------------------------------------------------------------------------------

\section{Data Flow}


The initial data flow of the kernels goes as follow, Fi

% Chapter Template
\chapter{Heterogeneous Performance Analysis and Practices} % Main chapter title

\label{Heterogeneous Performance Analysis and Practices} % Change X to a consecutive number; for referencing this chapter elsewhere, use \ref{ChapterX}

\lhead{Chapter 4. \emph{Heterogeneous Performance Analysis and Practices}} % Change X to a consecutive number; this is for the header on each page - perhaps a shortened title

While working with GPUs, new challenges emerge, such as how can we make the best use of the millions of threads using the GPU hardware. In the conventional CPU model, we have what is called linear or flat memory model, which appears to the programmer as a single contiguous address space. Furthermore, the CPU can directly address all the available memory, in other words, there is almost no efficiency penalty in creating global data, local data, or even access data that is located on an opposite memory location; All of this can be accessed as a contiguous block \cite{cook}. Meanwhile, on the GPU there are exceptions; there exists different memory hierarchies which dramatically change the performance. By allocating the optimal memory types, speedup and increase throughput can be accomplished. To ensure optimization, some analysis should be made, such as comparing latency, memory hierarchies and data bandwidth between CUDA kernels. Debugging of parallel code is possible  using the NVIDIA's Visual Profiler. This chapter demonstrates techniques, practices and methods to debug and analyze parallel process on the GPUs.

% techniques, programming models and hierarchies.
\section{Practices}

There are three rules for developing high performance GPGPU (General-purpose on the GPU) program, which are based on NVIDIAs GPU standards \cite{design}.

\begin{enumerate}
  \item Get the data on the GPU device and keep it there
  \item Process all the data en the GPU, give it enough work to do.
  \item Focus on data reuse within the GPU context, to avoid memory bandwidth limitations
\end{enumerate}

The GPUs are plugged into the PCI Express bus of the host computer. The PCIe bus has extremely slow bandwidth compared with the GPU. This is why it is important to store as much data as possible in the GPU and keep it busy, but as well minimize the data transfer from the host and back to the device. The process is illustrated in the Figure \ref{fig:PCI}. CUDA enables the GPU to carry out petaFLOP performance in a single device \cite{cook}. Moreover, the GPUs are fast enough to compute a large amount of data in parallel. To accomplish such high performance; each CUDA kernel needs to use all the available resources of the GPU, avoid wasting compute cycles and minimize thread branching.

\begin{figure}[htbp]
	\centering
		\includegraphics[width=0.72\textwidth]{Figures/PCI.png}
		\smallskip
	\caption[PCIe bus bandwidth]{PCIe bus and GPU bandwidth comparison \cite{cook}}
	\label{fig:PCI}
\end{figure}

The practices should be taken into consideration to identify the parts of code where it would be beneficial for improving GPU acceleration \cite{practices}.

\begin{figure}[htbp]
	\centering
		\includegraphics[width=0.45\textwidth]{Figures/apod.png}
		\smallskip
	\caption[GPU application practices]{GPU application practices \cite{practices}.}
	\label{fig:apod}
\end{figure}

\begin{description}

 \item{Asses} \hfill \\
 The first step is to locate the part of the code where the majority of the execution time occurs. The programmer can evaluate memory bottlenecks for GPU parallelization.
 \item{Parallelize} \hfill \\
 To increase parallelization from the original code, this could be done either by adding GPU-optimized libraries such as cuBLAS, cuFFT, or including more amount of parallelism exposure though the use of CUDA code.
 \item{Optimize} \hfill \\
The developer can optimize the implementation performance through a number of considerations, such as overlapping kernel executing, kernel profiling, memory handling and fine-tuning floating-point operations.
 \item{Deploy} \hfill \\
 Compare the outcome with the original expectation. Determinate the potential speedup by accelerating a given section. First a partial parallelization should be implemented before carrying out a complete change.
 \end{description}

\section{Performance Metrics}

There are many possible approaches for profiling CUDA code, but in all cases the objective is the same:  identify the kernel or kernels in which the application is spending most of its execution time and increase the throughput by a given kernel. Throughput is how many operations are completed per cycle.

\subsection{Timing}

Timing a launched kernel should be done on either the GPU or the CPU. However, the GPU and CPU are not synchronized, and events are able to block multiple threads at any given moment. Although it is necessary to synchronize the CPU thread with the GPU kernels launches. CUDA provides the required  functions to synchronize the CPU with the GPU calling immediately before starting the timer \cite{practices}. CUDA is able to handle timers within the GPU, which records times in a floating-point value in milliseconds. This is done with {\listf cudaEventRecord()}, just by including {\listf start} and {\listf stop } in the function inputs. Moreover, the timing is measure on the GPU clock, therefore, the timing is independent from the OS and CUDA \cite{cook}. Lastly, the timing results of the various stages of the simulation is found in chapter \ref{Optimization Results}.

\subsection{Bandwidth}

Bandwidth refers to the rate at which data can be transferred between host and device and vice-versa. The bandwidth is one of the most important factors for testing performance on the GPUs. Choosing the right type of memory could dramatically increase performance and bandwidth. There are two main bandwidth types to indicate performance: theoretical bandwidth and effective bandwidth. The theoretical bandwidth is based on the hardware specifications available by NVIDIA. The bandwidth is calculated using the following:

\vspace{0.8em}
\begin{center}
 {\listf theoretical bandwidth} = $\dfrac{({\listf clock rate} * (512 / 8.0 ) * 2.0)} { 10^{9}}$
\end{center}
\vspace{0.8em}

For example the NVIDIA GeForce GTX 280 uses DDR RAM with a memory clock rate of 1,105 Mhz and a 512-bit-wide memory interface

\vspace{0.8em}
\begin{center}
$\dfrac{(1107 * 10^6 * (512/8.0) * 2.0 )}{10^9}$ = $141.6$ Gb/sec
\end{center}
\vspace{0.8em}

The GTX 280 has a theoretical bandwidth of $141.6Gb/sec$. The effective bandwidth is calculated by timing specific program activities and by knowing how data is accessed by the application \cite{practices}.

\vspace{0.8em}
\begin{center}
{\listf effective-bandwidth} = $\dfrac{((Br - Bw) / 109.0 )}{time}$
\end{center}
\vspace{0.8em}

Where $Br$ is the number of bytes read per kernel, $Bw$ is the number of bytes written per kernel and  $t$ is the elapsed time given in seconds \cite{fortran}.

In practice the difference between theoretical bandwidth and effective bandwidth indicates how much bandwidth is wasted on accessing memory and calculations. If the effective bandwidth is low compared to the theoretical bandwidth, this is an indication that there is not enough work being done in the GPUs. In addition, there a several solutions: analyze the code to accomplish more parallelize instructions, execute more computational instructions on the GPUs, bind memory blocks, in other words pin the initial memory block on the CPU with the final memory  block on the CPU.

 In the chapter \ref{Optimization Results} we analyze the bandwidth and timing for each NVIDIA GPU used to optimize the application. However, bandwidth information is only available when transferring data from the CPU to the GPU or from the GPU to the CPU.

\section{Memory Handling with CUDA}

In this section four types of memory handling will be explained: global memory (device memory), shared memory, texture memory and constant memory. Figure \ref{fig:cores} illustrates physically the position of the different memory types inside the device chip.

\begin{figure}[htbp]
	\centering
		\includegraphics[width=0.95\textwidth]{Figures/cores.png}
		\smallskip
	\caption[Schematic cache hierarchy of a GPU]{The schematic cache hierarchy of a CUDA GPU with 4 Streaming Multiprocessors and 8 CUDA Cores each \cite{cook}.}
	\label{fig:cores}
\end{figure}

Global memory is very large in comparison to the shared memory, which is on the L1 cache. However, the global memory is far away from the registers and from the CUDA core locations. Moreover, the memory access is very slow in comparison to the shared memory \cite{cook}.

The Figure \ref{fig:memory} illustrates the five different memory types that are available in CUDA. But more interesting are the bandwidth penalty and the latency in computer cycles for each one of them. Moreover, different memory types can be used in different applications to maximize performance, hence memory usage. The Shared Memory is very limited so it cannot be handler across all situations. Furthermore, when implementing a wrong memory type on the device there are possibilities for latency penalties and bandwidth drop, instead of having a performance gain.
 
\begin{figure}[htbp]
	\centering
		\includegraphics[width=0.9\textwidth]{Figures/memory.png}
		\smallskip
	\caption[Different CUDA memory types]{Different memory type and penalties usage \cite{cook}.}
	\label{fig:memory}
\end{figure}

\subsection{Global Memory}

Understanding how efficiently to use global memory is essential part of CUDA memory management. Focusing on data reuse within the SM and caches is needed to avoid memory bandwidth limitations. Global memory on the GPU is designed to quickly stream memory blocks of data into the SM. However, global memory tends to be slow compared with there types of memory \cite{design}.

\begin{itemize}
\item Get the data on to the Device: keep it there.
\item Give the GPU enough workload: this uses all the resources available from the GPU.
\item Focus on data reuse within the GPGPU to avoid memory bandwidth limitations.
\end{itemize}

In other words the global memory resides in the device, and it should be anything from 1 byte to 8GB, depending on the GPU RAM available. Furthermore, the memory is visible to all the threads of the grid. Every thread at a given location is possible to read and to write as global memory. The memory is always allocated with the keyword \textit{cadaMalloc}. In addition, the global memory is only used by passing it to the kernel called the keyword \twoline global \twoline. Global memory is widely used for the current implementation \cite{design}.

\subsection{Shared Memory}

The CUDA C compiler treats variables differently than a typical c language variable. The compiler creates a copy of the variable for each block that is launched on the GPU. Now every thread in that block has access to the memory, hence, shared memory. This memory resides physically on the GPU, because the memory is very close the cache. The latency is typical very low \cite{example}. One thing comes to mind, if the threads can communicate with others threads, there should be a way to synchronize all the threads. A simple case should be if thread A writes a value into the shared memory, and Thread B wants to be accessed we need to synchronize. When thread A is finished writing then thread B can access it. This is typical case when shared memory with synchronize thread is needed \cite{cook}.

Shared memory is much faster to access than global memory, and essentially it is like a local cache for each thread of a block. While  the shared memory is limited to 48K a block, the global memory is the amount of DRAM on the device. The duration of the shared memory on the device is the lifetime of the thread block. Using \twoline shared \twoline in-front of the data type will innovate shared memory.

Shared memory is widely used for applications where the kernels access a great amount of global memory. In addition, using shared memory eliminates the use of clock cycles per kernel which increases performance on a single kernel call. For the current application we used extensively shared memory, eliminating the use of global memory. More information about the process ca be found in Chapter \ref{Optimization Results}.

   
\subsection{Constant Memory}

Constant memory is an excellent way to store and broadcast read-only data to all the threads on the GPU. One thing to keep in mind is that the constant memory is limited to 64KB \cite{design}. A simple analogy is the {\listf \#define} or {\listf const} attribute in the C++ programming language. The variable performs like a value that cannot be modified. On CUDA this is exactly the same, and the value can only be read and not written. Furthermore, the value will not change over the course of a kernel execution and only the host can write the constant memory \cite{example}. Constant memory is able to reduce calls for static global memory. In chapter \ref{Optimization Results} we discuss improvements speedup improvements by using this type of memory.
\subsection{Texture Memory}

Similar to constant memory, texture memory is another variety of read-only memory that can improve performance and reduce memory traffic when reads have certain access patterns. Traditionally texture memory is used for computer graphics applications, but it can also be used for HPC. The main idea of this read-only memory is that threads are likely to read from address 'near' the address of the nearby threads \cite{example}.

\begin{figure}[htbp]
	\centering
		\includegraphics[width=0.52\textwidth]{Figures/texture.png}
		\smallskip
	\caption[Texture memory]{Mapping of threads into a two dimensional array of texture memory \cite{hwu}.}
	\label{fig:texture}
\end{figure}

The texture memory in a form works like the GPU graphics texture: when you want to use the texture bind with some sort of data is necessary and when you finish using it, unbind the texture from the data. The usage can be summarized in the following list:

\begin{itemize}
\item Allocate global memory in the Host.
\item Create texture reference and bind it to memory object.
\item On the device obtain the reference from the texture.
\item  Use Texture memory operations on the device
\item  When the work is done on the texture, unbind the texture reference on the host.
\end{itemize}

The texture memory is not used on the current implementation, for obvious reasons: it is a read only memory. For the numerically methods we need to constantly read and write blocks of memory.

\subsection{Thread Synchronization}

Thread Synchronization refers to orderly execute thread operations. For efficiency, a pipeline can be created by queuing a number of kernels to keep the GPGPU busy for as long as possible. Furthermore, some form of synchronization is required so that the host is able to determine when the kernel or pipeline has been completed \cite{design}. Commonly used synchronization mechanisms are:

\begin{itemize}
  \item Explicitly calling {\listf cudaThreadSynchronize()}, which acts as a barrier causing the host to stop and wait for all queued kernels to be completed.
  \item Performing a blocking data transfer with {\listf cudaMemcpy()} as {\listf cudaThreadSynchronize()} is called inside {\listf cudaMemcpy()}.
\end{itemize}

The basic unit of work on the GPU is a thread. It is important to understand from a software point of view that each thread is separate from every other thread. Every thread acts as if it has its own processor with separate registers and identity. It will wait for all threads to finish their job \cite{design}.

Thread synchronization is also possible inside kernel calls. The idea is the same: the kernel will wait until all the threads have completed their task. When more threads are synchronized, they schedule more work, hence, better performance and more workload. Thread synchronization is generally used when loading data into shared memory. The implementation of such process is in Chapter \ref{Optimization Results}, section optimizations. 

\section{Concurrent Kernels}

Kernels are executed in a sequential form with parallel instructions. In addition, with CUDA's streams is possible to launch several kernels in parallel, in other words, overlap kernel in the same launch sequence. Figure \ref{fig:streams} illustrates this.

\begin{figure}[htbp]
	\centering
		\includegraphics[width=0.68\textwidth]{Figures/streams.png}
		\smallskip
	\caption[Concurrent kernels]{Overlapping kernel execution using CUDA streams.}
	\label{fig:streams}
\end{figure}

A stream in CUDA is a sequence of operations that execute on the device in the order in which they are issued by the host code. Every kernel is launched on the default stream zero. Hence, to overlap kernel execution, non-default streams should be used for every kernel launch. To accomplish concurrent kernels, streams should be pinned to a non-default stream (non zero)\cite{hwu}. 

Using two or more CUDA streams, we can allow the GPU to simultaneously execute a kernel while performing a copy between the host and the GPU. However, we need to be careful about two issues. First, the host memory involved needs to be allocated. Since we will queue our memory copies, we need to synchronize those copies. Second, we need to be aware that the order in which we add operations to our streams will affect our capacity to achieve overlapping of copies and kernel execution. The general guideline involves a breadth-first, or round robin, to assign work and queue work to the kernels \cite{example}.

The order of kernel executing affects the operations of the streams, and moreover, the application performance. In the current application, we carefully examined the order of kernel executing. More about how to implement concurrent kernels for the simulation is found in Chapter \ref{Optimization Results}.

\section{Kernel Analysis}

 Kernels are the essential part of CUDA programming;, threads are launched automatically throughout each thread per blocks of the device. Furthermore, millions of threads execute the same code in parallel. However, the parallel code can be bound by three factors memory, compute and latency \cite{cook}.

\begin{description}

 \item{Memory Bandwidth Bound} \hfill \\
  Refers to the application's limitations for memory access. Most GPUs cards have 1GB-6GB of memory, which is used to process the data on the GPU. Different solutions are: reuse data, use different GPU memory types, or implement a multi-GPU approach to increase the memory.

  \item{Compute Bound} \hfill \\
Refers to the computation time execution: in other words, calculations done on the device, under the assumption there is enough memory for the calculations. Therefore, it is the number of operations per cycle in the kernel. Theoretical bandwidth vs. effective Bandwidth can measure performance for a compute-bound Kernel. Therefore, it is possible to increase the FLOPS per device.

 \item{Latency Bound} \hfill \\
 Is one whose predominate stall and it is due to memory fetches. This is actually the saturating the global memory, or any type, but it still has to wait to get the data into the kernel. Physically, it is data being sent from one part of the device to the other. Moreover, depends on the time required to perform an operation. It is counted in cycles of operations. A way to reduce the latency is to increase the number of parallel instructions (more  calls per thread), in other words more work per thread and fewer threads. However, this is not always possible.
 \end{description}
 
Depending on the problem, the application can be bound by the previous three factors. In the next chapter we will explain how and why the current implementation is bounded by memory, compute and latency.

\section{Hardware constraints}


The hardware capabilities, limits how many threads per block a kernel launch is able to have, but as well as the version of CUDA that the hardware is able to execute. The compute capabilities of a device represents by a version number, called -"SM version" or CC for short. The version number identifies the features supported by the GPU hardware and is used by the applications at runtime to determine which hardware features and/or instructions are available on the present GPU \cite{tool}. 

Another inefficiency, that can cause low performance in the CUDA application, is the number transfers memory calls between the CPU and the GPU. The GPU communicates with the CPU via a \textit{PCIe} bus as mentioned before. In addition, all of the massive FLOPS per second that are computed on the GPU are not able to be sent back to the CPU, because, of the physical connection between the GPU and CPU. Ideally, the GPU should have a large workload as possible before returning data back to the CPU. However, this is not always possible, a technique to increase more throughput is to pin/bind the memory in the host. Another method is to send as much workload as possible in a single kernel call and by using the maximum the GPU hardware capabilities \cite{practices}. For the current implementation, CPU and GPU are relatively low, only a few times communication is done by the device and the host.

\subsection{Thread Division}

There are several hardware limitations in how many threads per block a kernel can handle. Launching a kernel with the hardware constraints of the device will only ensure us that the kernel will actually be executed on the device, Nonetheless, not 100$\%$ optimal and the results can be incorrect. Furthermore, it is necessary to launch kernels with the right amount of threads per block based on the hardware settings. The block size will determine how fast the kernel will execute. However, not the biggest block will run faster, relays on the problem and the data set. By benchmarking the application, it is possible to find the optimal configuration that best fits the problem. One thing to keep in mind is that thread blocks should be a multiple number of SMs. With this idea, it is possible to obtain optimal thread block configuration. Review Chapter \ref{Optimization Results} for the optimal thread configuration for the current simulation. The optimal number of threads per block did not occur on the maximal available threads per block of the devices.

\section{Visual Profiler}

It is a challenging task to keep track of each individual thread even more, a million threads. It becomes difficult to debug highly parallel applications. The NVIDIA's Visual Profiler is a profiling tool that can be used to measure performance and find potential opportunities for optimization in order to achieve maximum performance on the GPUs. The Profiler provides metrics in the form of plots and graphs, which illustrate instances of the GPU, such as kernel calls, data transfer, kernel executing time, memory dumps and others. See Figure \ref{fig:visualgraph}.

\begin{figure}[htbp]
	\centering
		\includegraphics[width=1.0\textwidth]{Figures/visualgraph.png}
		\smallskip
	\caption[Visual Profiler metrics]{Profiler provides optimization metrics necessary to improve the application.}
	\label{fig:visualgraph}
\end{figure}

NVIDIA's profiling tools comes in various ways; a standalone profiler through the visual profiler compiler nvvp, integrated in a GUI NSight Eclipse Edition as NSight command (Visual Profiler), and as a command-line profiler though nvprof command. Each one has its disadvantages and advantages. The command-line profiler is useful for remotely access, where a GUI is not available, while the NSight can show graphs, plots and timeline of the application. The Profiler support CUDA applications as well as openCL applications. However, there are exceptions. 

The Visual Profiler, by default, will execute the entire application, nonetheless typically only some parts of application only need performance optimization. This enables to determine kernels, code where critical performances is needed. The common situation where profiling a region of the application is helpful \cite{tool}.

\begin{itemize}
  \item Analyze data initialization and movement in the CPU and GPU, as well as evaluating CUDA calls.
  \item The application operates in phases, where an algorithm operates throughout each region. The application can be optimized independently from other phases of the code.
  \item The application contains algorithms that operate though a large number of iterations. In this case it is possible to collect data from a portion of the iterations.
\end{itemize}

The Visual Profiler provides a step-by-step optimization guidance, where it is possible to evaluate the GPU usage, to examine individual kernels and to analyze timeline of the application which the profiler shows memory movements and usage, CUDA calls, number of threads and performance. Figure \ref{fig:visual01} shows that each kernel has its own percentage of execution time of the overall application \cite{practices}.

\begin{figure}[htbp]
	\centering
		\includegraphics[width=0.9\textwidth]{Figures/pofiler.png}
		\smallskip
	\caption[Visual Profiler timeline and stream process]{Visual Profiler kernel execution, and timeline execution.}
	\label{fig:visual01}
\end{figure}

\subsection{Profiler Kernel Report}

The profiler will execute several times the application for it to collect data from each kernel. This enables it to precisely optimize phases of the application\cite{example}. The profiling tools can verify how long the application spends executing each kernel as well the number of used blocks and threads. Through this it is possible to obtain various memory throughput measures, like global load throughput and global store throughput. They indicate the global memory throughput requested by the kernel and therefore corresponding to the effective bandwidth mentioned in the last section.

As we know, the profiler executes the application several times to collect data about each kernel. The information obtained by each kernel can be summed-up in-to a report that can be exported in a pdf file, which has the following information.

\begin{enumerate}
  \item \textbf{Compute, Bandwidth, or Latency Bound} \hfill \\
      The performance determines if the kernel is bounded by computation, memory bandwidth, or instructions/memory latency. It shows how it is limiting the performance respectively.
  
  \item \textbf{Instructions and Memory Latency} \hfill \\
Both limit the performance of a kernel when the GPU does not have enough work to keep busy. The performance of latency-limited kernels can often be improved by increasing occupancy. Occupancy is a measure of how many warps the kernel has active on the GPU, relative to the maximum number of warps supported by the GPU.
  
  \item \textbf{Compute Resources} \hfill \\
GPU compute resources limit the performance of a kernel when those resources are insufficient or poorly utilized. Compute resources are used most efficiently when instructions do not overuse a function unit. 
  \item \textbf{Floating-Point Operation Counts} \hfill \\
  Floating-point operations executed by the kernel can be either single precision or double precision.
  
  \item \textbf{Memory Bandwidth} \hfill \\
  Memory bandwidth limits the performance of a kernel when one or more memories in the GPU cannot provide data at the rate requested by the kernel.
\end{enumerate}

The profiling report was used in the current implementation to optimize the CUDA code. More about the results can be found in Chapter \ref{Optimization Results}.

\subsection{Collect Data on Remote System}

As mention before, it is possible to collect data from a remote system where a GUI is not available, using the command-line nvprof. Remote profiling is the process of collecting profile data from a remote system that is different than the host system at which that profile data will be viewed and analyzed. Once all the profiling results are collected it is possible to access the information using a local Visual profiler. It enables a GUI and more compressive information about the application. There are two ways to perform a remote profiling. To use nvvp remote profiling you must install the same version of the CUDA Toolkit on both the host and remote systems. It is not necessary for the host system to have an NVIDIA GPU \cite{tool}. For the current application, remote profiler was used. However, the server did not have an external monitor or virtual. Therefore, it was not possible to obtain all the profiling analysis, thus we used a local laptop for profiling.

\vspace{4.0em}

Finally, the chapter gives an overview of practices and performance studies for GPGPU and a better understanding of the hardware and memory management on the GPU. In addition, the hardware limitation are able to determinate the best usage of the GPUs. The NVIDIA's profiling tool is useful to analyze different stages of our application. Therefore, is possible to determinate elements of the CUDA code where is better to optimize from others, thus, gain an improvement in performance and reduction of executing time.


% Chapter Template

\chapter{Optimization Results} % Main chapter title

\label{Optimization Results} % Change X to a consecutive number; for referencing this chapter elsewhere, use \ref{ChapterX}

\lhead{Chapter 5. \emph{Optimization Results}} % Change X to a consecutive number; this is for the header on each page - perhaps a shortened title


In this chapter we obtain the results of the CUDA code implementation launched on a single GPU device. The tests were performed on various GPUs architectures. The application is analyzed on various stages using NVIDIA's Visual Profiler. In addition, the CUDA kernels were evaluated in performance, execution time, occupancy and concurrent kernels. Furthermore, the results, are analyzed and optimized using the schemes from Chapter \ref{Heterogeneous Performance Analysis and Practices}. The code is executed remotely on the supercomputer Piritakua at the Department of Multidisciplinary Studies Yuriria, University of Guanajuato. The last section is the overview of all the optimizations results achieved on the simulation.

\section{Supercomputer Piritakua}

The experiments are carried out using the supercomputer Piritakua. The massive GPU cluster was design and built by Dr. Claudio from the University of Guanajuato at Yuriria's Multidisciplinary Studies. The cluster is located at Yuriria, a small town in the center of Mexico. Piritakua at its front-end has an eight core Intel Xeon at 2.4 Ghz, at the back-end several GPU are connected. All of GPUs CUDA version 5.0 was installed. Furthermore, the GPUs node are; one NVIDIA Tesla K20, two Tesla M2070 and a GeForce GTX 580 

The cluster has the CentOS ( Community Enterprise Operating System) 64 bits as operating system. The OS is a free operating system and one of the most popular GNU$ \ $Linux distribution for web servers and as well is supported by RHEL (Red Hat Enterprise Linux) \cite{centos}. The specifications of cluster are \ref{tab:cpus}.

\begin{table}[h]
\centering
\begin{tabular}{ | p{7.1cm}  | l | l | l |}
  \hline
  Processor & Number & Cores & RAM  \\
  \hline
  Server Dell Intel(R) Xeon(R) E5620 2.4 GHz & 1 & 8 & 12 GB \\
  \hline
  Server HP Proliant SL 350s Gen3 Intel(R) Xeon(R) X5650 2.67 GHz & 2 & 24 & 32 GB \\
  \hline
   Server HP Proliant SL 250s Gen8 Intel Xeon E5-2670 2.60 GHz & 3 & 48 & 104 GB \\
   \hline
  \end{tabular}
      \caption{CPU technical specifications}
  \label{tab:cpus}
  \end{table}

The host code was executed on only two types of CPUs. On an eight core Intel i7-3630QM and on a high-end eight core Intel Xeon CPU E5620. In addition, the Xeon was used in all the simulation test, except when testing on the GeForce 670mx. Lastly, the code was executed on laptop to show the performance comparison between a lightweight GPU and a server based GPU.

The device code was executed using the CUDA Toolkit 5.5 version. The main advantage of the 5.5 version are improvements in MPI(Message Passing Interface) and HyperQ. The MPI implementation was not used for the current simulation. However, we obtain benefits from using the HyperQ when applying concurrent kernels to the implementation. 

When accessing Piritakua remotely is possible to use all the GPUs nodes available on the cluster. The specifications of the GPU connected to the back-end are as follow, CC stands for compute capability.

\begin{table}[h]
\centering
  \begin{tabular}{ |  l  |  l  |  l  |  l  |  l  | l | l | l |l | }
    \hline
    Model & Core& RAM& DP GF& SP GF& Bandwidth& GHz& CC & Power\\
    \hline
    Tesla K20m & 2496 & 5GB & 1,170 & 3,520 & 208GB/s & 0.73 & 3.5 & 225W \\
   \hline
    Tesla M2070 & 448 & 6GB & 515 & 1,030 & 150GB/s & 1.15 &  2.0 & 225W\\
   \hline
     Tesla C2050 & 448 & 2.5GB & 512 & 1,030 & 144GB/s & 1.15  & 2.0 & 238W \\
   \hline
      GeForce 580 & 512 & 1.5GB & 520 & 1,154 & 192.2GB/s & 1.5 & 2.0 & 244W \\
   \hline
   GeForce 670mx & 960 & 3GB & 520 & 1,154 & 67.2GB/s & 0.6 & 3.0 &  - \\
   \hline
  \end{tabular}
    \caption{GPU technical specifications}
  \label{tab:gpus}
  \end{table}
  
  The code was launched on all Piritakua's GPUs and on the GeForce GTX 670m. The "m" stands for the mobil graphic cards. In addition the 670m card is design for less power usage, but high graphics power, it has more cores than some Tesla models, However, they have less Bandwidth than standard versions. The 670m card was used as comparison between laptop GPUs and high-end desktop/servers GPUs.
     
\subsection{Architecture Differences}

  NVIDIA's GPU are constantly being improved over the years, nonetheless, most programming paradigms stayed the same. The are currently three main architectures; Fermi, Kepler and Maxwell. For example, a streaming processor can now handle 2048 threads at a time, but the maximum block size stayed at 1024. The results in a 100$\%$ theoretical occupancy for block sizes of 1024 compared to 66$\%$ of Fermi type. Another example is the use of Shared Memory. Maxwell has 64KB dedicated Shared Memory. The maximum amount of Shared Memory per Block is 48KB for all three architectures \cite{hoermanngpu}.
  
  There are two GPU architectures where the implementation was launched, the Fermi and the Kepler. The Tesla K20m and the GeForce 670mx are based on the ``Kepler'' GPU architecture. The Tesla M2070, M2050 and the GeForce GTX 580 on the Fermi architecture. The Kepler architecture newer than the Fermi. More information about the architectures in the table \ref{tab:arch}. The Maxwell architecture was not used for the current simulation, however, is showed for future reference.
  
\begin{table}[h]
\centering
  \begin{tabular} { | l | l  | l | l | l  |  l  | l |}
    \hline
    Name & \multicolumn{2}{|c|}{Fermi} & \multicolumn{2}{|c|}{Kepler} &  \multicolumn{2}{|c|}{Maxwell} \\
    \hline
    Compute Capability & 2.0 & 2.1 & 3.0 & 3.5 & \multicolumn{2}{|c|}{5.0}\\
   \hline
    Single Precision Operation per Clock/SM & 32 & 48 & \multicolumn{2}{|c|}{192} & \multicolumn{2}{|c|}{128}\\
   \hline
    Double Precision Operation per Clock/SM & $4/16^1$ & 4 & 8 & $8/64^2$ & \multicolumn{2}{|c|}{$1^3$}\\
   \hline
    Max Number of Threads per SM / SM & \multicolumn{2}{|c|}{16} & \multicolumn{4}{|c|}{32}\\
   \hline
    Max Number of Registers per Thread/SM & \multicolumn{3}{|c|}{1536} & \multicolumn{3}{|c|}{2048}\\
   \hline
       Max Number of Threads per Block & \multicolumn{6}{|c|}{1024}\\
   \hline
   Active Thread Blocks per SM / SM & \multicolumn{2}{|c|}{8} & \multicolumn{2}{|c|}{16} & \multicolumn{2}{|c|}{32}\\
   \hline
   Max Warps per Multiprocessor/ SM & \multicolumn{2}{|c|}{48} & \multicolumn{4}{|c|}{64}\\
   \hline
   Registers / SM & \multicolumn{2}{|c|}{32K} & \multicolumn{4}{|c|}{64K}\\
   \hline
   Level 1 Cache & \multicolumn{2}{|c|}{16/48 KB} & \multicolumn{2}{|c|}{16/32/48 KB} & \multicolumn{2}{|c|}{64 KB}\\
   \hline
   Shared Memory / SM & \multicolumn{2}{|c|}{16/48 KB} & \multicolumn{2}{|c|}{16/32/48 KB} & \multicolumn{2}{|c|}{64 KB}\\
   \hline
   Warp Size & \multicolumn{6}{c|}{32}  \\
   \hline
  \end{tabular}
  \caption{GPU Architecture Specifications}
  \label{tab:arch}
  \end{table}
 
\section{Optimization}

The cluster has two  main architecture types Fermi and Kepler. Furthermore, the initial results of 1.00x speedup were tested on a Kepler architecture, the GeForce GT 650M. The graph \ref{fig:iniresults} illustrates the executing time for all the GPUs in the server.

We used NVIDIA's Visual Profiler to obtain kernel metrics of the Tesla K20m, table \ref{tab:nvprof}. The output is organized by kernel importance and performance during the simulation. The Laplacian kernel evaluation; {\listf glaplacinay}, {\listf gLaplacianx} and  {\listf gLaplacianYBoundaries} uses upto 44.37$\%$ of the overall simulation. The $gsolution$ kernel, which solves Zhang-Li model \ref{eq:zhang} consumes up-to 14.04$\%$. The RK4 integration only exhaust a minor part of the overall simulation. However, the {\listf gSolution}, {\listf gsdExchange} and Laplacian calculation are part of the RK4 integration, which overall is about 99$\%$.

The throughput was not review on the current application. Since only two stages of the simulation transfer of the CPU data occurs, on the initial stage where CPU data is sent to the GPU, and the final stage where is sent back to CPU.

 The optimization focus is to give the GPUs as much work as possible, using at the fullest the GPU hardware capabilities. In addition, reducing the overall performance time of each kernel, eliminating the computational hover-head process on the highest consumed kernel \ref{tab:nvprof}.
 
\begin{table}[h]
\centering
  \begin{tabular} { | l | l | l | l | l | l | l |}
    \hline
    Time$\%$& Time & Calls & Avg & Min & Max & Kernel \\
    \hline
    23.50 & 3.6s & 26521 & 137.5us & 96.0us & 597.1us& {\listf gLaplaciany} \\
    \hline
    17.04 & 2.6s & 26521 & 99.7us & 57.0us & 561.1us& {\listf gSolution} \\
    \hline
    16.75 & 2.6s & 26522 & 98.0us & 62.8us & 400.6us& {\listf  gLaplacianx} \\
     \hline
      13.37 & 2.0s & 26522 & 78.2us & 40.8us & 453.8us& {\listf gsdExchange} \\
      \hline
    7.22 & 1.1s & 26522 & 42.2us & 23.4us & 326.0us & {\listf gsfRelaxation} \\
       \hline
    6.22 & 965.2ms & 6630 & 145.6us & 79.2us & 722.6us & {\listf gTerm4RK4}  \\
       \hline
    4.12 & 640.3ms & 26522 & 24.1us & 21.8us  &138.7us & {\listf gLaplacianYBoundaries} \\
       \hline
    3.41  & 529.2ms & 6630 & 79.8us & 41.6us  & 478.8us & {\listf gTerm2RK4} \\
       \hline
    3.36 & 520.8ms & 6630 & 78.5us & 41.5us & 372.2us & {\listf gTerm3RK4} \\
       \hline
    3.35 & 519.5ms & 6631 & 78.3us & 41.1us & 372.2us & {\listf gTerm1RK4} \\
   \hline
  \end{tabular}
  \caption{Kernel time executing and on the Tesla K20}
  \label{tab:nvprof}
  \end{table}
 
  \begin{figure}[htbp]
	\centering
		\includegraphics[width=1.0\textwidth]{Figures/gpu_initial.png}
		\smallskip
	\caption[Initial GPU results]{Initial implementation benchmarking on several different GPUs nodes, with a initial speedup of 1.0x}
	\label{fig:iniresults}
\end{figure}

The figure \ref{fig:iniresults} illustrates the GeForce GTX 580 is the card with the least amount of execution time, and the Tesla K20m the fastest amongst the Tesla Cards. 


The following sections are the optimization techniques and methods applied to the application. In addition, comparing the performance between the initial implementation and the modified versions. The optimization is breakdown into five steps; Branching, Occupancy, Concurrent Kernels, Shared Memory and Structure of Arrays. Branching, refers how kernels and threads are executed in the application. Occupancy, number of threads per devices being used. Concurrent Kernels, execution several kernels at once. Shared Memory, using as much shared memory as possible. Finally, Structure of Arrays, modification of the memory allocation in the device.

 \subsection{Branching}
 
 CUDA follows the Single Instruction Multiple Thread architecture, meaning, threads execute the same code. Each thread is able to operate on its own data and has its individual address counter. Moreover, threads are free to use a different path. Each thread launches the same operation at the same time. However, they have to wait for all the threads in the kernel to finish their task. In other words, some threads can finish their job before a groups of threads are executing their tasks. Therefore, a thread within a warp/block branches differently the other threads get deactivated \cite{hoermanngpu}. The method is described in the following code \ref{lst:branch} and  illustrated in the figure \ref{fig:threads}.

\begin{lstlisting}[language=C++, label={lst:branch}, caption={Branching threads}]
__global__ void kernel(int* out){
idx = threadIdx.x;
int result;
if(idx == 0){
	result = foo();
} else {
	result = bar();
	out[idx] = result;
}
\end{lstlisting}

\begin{figure}[htbp]
	\centering
		\includegraphics[width=0.6\textwidth]{Figures/threads.png}
		\smallskip
	\caption[he execution flow]{The execution flow of a branching code, with warp size 8. Black arrows are active threads, and the grey ones are disabled.}
	\label{fig:threads}
\end{figure}

The branching problem occurred in the section where boundary condition for Laplacian was being analyzed \ref{lst:branchlap}. Only a single kernel was used to check the boundary condition. In addition, a bottleneck occurred. The implementation gets the job done with only one kernel. However, a minor part of the threads are only working, which is a waste of computation resources and energy.

\begin{lstlisting}[language=C++, label={lst:branchlap}, caption={Branching problem in the Laplacian boundary condition evaluation}]
__global__ void glaplaciany(...); //Compute Laplacian in Y direction
__global__ void glaplacianx(...); //Compute Laplacian in X direction

__global__ void glaplacianyboundaries(...){
    if (i > 1 && i < NX + 2 && j == 0){
     	// Update Laplacian Boundaries
    }
    else if (i > 1 && i < NX + 2 && j == 1){
  		// Update Laplacian Boundaries
  	}
    else if (i > 1 && i < NX + 2 && j == NY - 2){
        // Update Laplacian Boundaries
    }
    else if (i > 1 && i < NX + 2 && j == NY - 1){
        // Update Laplacian Boundaries
    }
}
\end{lstlisting}

To solve the branching issue, we include more work on the laplacian boundary condition kernel. The new kernel evaluates the boundary condition in a single kernel. Therefore, eliminating branching threads, more importantly, reducing global memory calls (code \ref{lst:newcde}).

\begin{lstlisting}[language=C++, label={lst:newcde}, caption={More workload on a single kernel execution}]
__global__ void glaplacianyboundaries(...){
    if (i > 1 && i < NX + 2 && j == 0){
     	// Update Laplacian Boundaries
    }
    else if (i > 1 && i < NX + 2 && j == 1){
  		// Update Laplacian Boundaries
    }
    else if (i > 1 && i < NX + 2 && j == NY - 2){
        // Update Laplacian Boundaries
    }
    else if (i > 1 && i < NX + 2 && j == NY - 1){
        // Update Laplacian Boundaries
    }
    glaplaciany(...); //Compute Laplacian in Y direction
	glaplacianx(...);  //Compute Laplacian in X direction
}
\end{lstlisting}

The technique was applied to all parts of the code. Therefore, eliminated inactive threads. Moreover, activating threads for more computational process. The technique increased the occupancy percentage of active threads within the kernels. The results of the modified version please read the final section of the chapter.

\subsection{Concurrent Kernels}

Initially, each kernel was launched on the default steam zero. Therefore, every kernel was consequently launched in a serial way. The figure \ref{fig:streams} illustrates such results using the NVIDIA's Visual Profiler. Each kernel that is being launched is not able to run simultaneously. Because, each kernel needs previous data to compute the next data. In other words, the kernels are not independent from each other. Therefore, we change the implementation to be able to launch parallel kernels.

\begin{figure}[htbp]
	\centering
		\includegraphics[width=1.0\textwidth]{Figures/ini_steams.png}
		\smallskip
	\caption[Initial Streams]{Kernels running on the default Stream zero.
}
	\label{fig:streams}
\end{figure}

 Kernels by default cannot run in parallel with others kernels. Furthermore, CUDA doesn't provide an automatic parallel kernel executing. In addition, the programmer needs to tell the CUDA compiler that some portion of the code or kernel should be run in parallel. However, the compiler does not always know when to use concurrent kernels, it depends on the hardware capabilities and as well the number of threads per block and the number of SM available. If the compiler finds available space to run another kernel simultaneously, it will do so. 

For example, the {\listf gsolution} \ref{lst:concurrent} kernel computes the Zhang and Li model for x, y, z coordinates, which extensively uses the global memory of the device. To accomplish concurrent kernels, the streams should be able to access memory blocks that are pinned to a specific stream. Therefore, each memory block corresponding to x, y, z coordinate are mapped to three independent streams. Furthermore, all the matrices corresponding to the coordinate x are mapped to the stream 1, coordinate y to stream 2 and coordinate z to stream 3.

\begin{lstlisting}[language=C++, label={lst:concurrent}, caption={Evaluation of x, y, z coordinates of the Zhang and Li model in a single kernel.}]
deltamX[index] = sfrelaxX[index] + sdexX[index] + laplX[index] - smX[index];
deltamY[index] = sfrelaxY[index] + sdexY[index] + laplY[index] - smY[index];
deltamZ[index] = sfrelaxZ[index] + sdexZ[index] + laplZ[index] - smZ[index];
\end{lstlisting}
 
The CUDA code \ref{lst:concurrent} is divided into a single kernel \ref{lst:consingle}. In addition, the new generic kernel is launched parallel with the others kernels. Instead of running one big kernel, three individual kernels are launched simultaneously. Dividing each kernel is now possible to implement shared memory through each kernel, otherwise wasn't possible. 

\begin{lstlisting}[language=C++, label={lst:consingle}, caption={Evaluation of individual coordinates of the Zhang and Li model}]
int i = blockIdx.x * blockDim.x + threadIdx.x + 2;
int j = blockIdx.y * blockDim.y + threadIdx.y;
int index = j * NXCUDA_CONST + i;

if (i > 1 && i < NX + 2 && j >= 0 && j < NY)
	deltam[index] = sfrelax[index] + sdex[index] + lapl[index] - sm[index];
\end{lstlisting}

This same method was applied to every kernel that was possible to separate into three kernels calls. Some kernels were not viable to be separated, such as the cross product. Because, the cross product uses pinned memory block from the other streams. The figure \ref{fig:concurrent} illustrates the results of concurrent kernels in the Tesla K20.

\begin{lstlisting}[language=C++, caption={Evaluate Zhang and Li model.}]
for (int i  = 0; i < 3; i++)
	gsolution<<<blocks, threads, 0, stream[i]>>>(spinAccXYZ[i]->getDev_deltam(),
												 spinAccXYZ[i]->getDev_sfrelax(), 
												 spinAccXYZ[i]->getDev_sm(), 
											 	 spinAccXYZ[i]->getDev_sdex(),
											 	 spinAccXYZ[i]->getDev_lapl());
\end{lstlisting}

\begin{figure}[htbp]
	\centering
		\includegraphics[width=1.0\textwidth]{Figures/concurent.png}
		\smallskip
	\caption[Streams kernels Tesla K20]{Concurrent kernels in the Tesla K20 using NVIDIA's Visual Profiler.}
	\label{fig:concurrent}
\end{figure}

In chapter \ref{Implementation of Domain Wall Dynamics under Nonlocal STT} we mentioned that the input data set 480 x 120 is mapped to a CUDA square grid of 512 x 128. Because the mapping of the extra threads, is possible to execute a portion of kernels concurrently.

Concurrent kernels demonstrate a very promising technique to achieve a huge performance increment in the current simulation. In theory is possible to have multiple kernels executing at the same. However, there are some downsides to the implementation; correctly synchronize kernels, waiting time and hardware resources are among the  problems \cite{practices}. The timeline of the application \ref{fig:waittime} illustrates the waiting time between kernels execution. However, the waiting time are very small time steps between 0.01ms and 0.01ms, but waiting time occurs for each step of the RK4, appears approximate 45,00 times. Furthermore, branching the kernel execution process should eliminate the issue.

\begin{figure}[htbp]
	\centering
		\includegraphics[width=1.0\textwidth]{Figures/waittime.png}
		\smallskip
	\caption[Waiting time in concurrent kernels]{Waiting time between each concurrent kernel execution}
	\label{fig:waittime}
\end{figure}


\subsection{Shared Memory}

Shared memory is faster than global memory(read Chapter \ref{Heterogeneous Performance Analysis and Practices} for more reference), however, shared memory is very limited. To be able to implement shared memory in the kernels, we needed the kernels to be separated in their x, y and z coordinate, as mentioned in the previous section. In addition, this allows us to implement shared memory across each kernel, otherwise wouldn't be possible. Shared memory was applied in all the kernels were global memory was used extensively. Eliminating the number clock cycles per thread.

The idea behind shared memory is to reduce the amount of global memory calls, which has about 400-600 clock cycles, while the shared memory only 1-32 clock cycles \ref{fig:memory}. The shared memory implementation is accomplish by allocating the data from the thread block into a temporary array, in other words the shared memory. In addition, the kernel is able to performed calculations on the temporary array and write the values onto the global memory. The implementation code is illustrated in the listing \ref{lst:shared}. There is no guaranty that threads will execute at the same order. Using {\listf \_\_syncthreads()} will wait until all threads have completed their task, in this case loading global memory into the shared memory array. Chapter \ref{Heterogeneous Performance Analysis and Practices} section thread synchronization, has more information about thread synchronization and shared memory. Once all the operations on the shared memory array are finish. The final part is to write the shared memory values back to the the global memory.

\begin{lstlisting}[language=C++, label={lst:shared}, caption={Shared memory}]
int i = blockIdx.x * blockDim.x + threadIdx.x;
int j = blockIdx.y * blockDim.y + threadIdx.y;
int index = j * NXCUDA_CONST + i;

if (i > 1 && i < NX + 2 && j >= 0 && j < NY){
     int cacheIdx = threadIdx.y * blockDim.x + threadIdx.x;
     __shared__ double deltamS[THREADS_SHARED * THREADS_SHARED];

	 //load memory into shared memory
     deltamS[cacheIdx] = operationGlobal(globalMemory);
     __syncthreads();

	 //copy back the shared memory to global memory
     deltam[index]  = deltamS[cacheIdx];
}
\end{lstlisting}

To calculate the Laplacian, we need to access a great amount of global memory, therefore, located near the value of interest. In this case a region of 4x4 grid. The figure \ref{fig:shared} illustrates what part of the block is used for allocating shared memory and global memory. The global memory is used for the boundary conditions of the block, while the shared memory for all the values inside the block.

\begin{figure}[htbp]
	\centering
		\includegraphics[width=0.4\textwidth]{Figures/shared.png}
		\smallskip
	\caption[Shared Memory Strategy]{Shared Memory Strategy for Laplacian evaluation }
	\label{fig:shared}
\end{figure}

The code \ref{lst:lapsh} demonstrates to how calculate the Laplacian from the equation \ref{eq:nn} with the implementation of shared memory. First we load all the global memory into a temporary array, the shared memory. Then we performed the calculation on the shared memory as mentioned before. Lastly return data to the global memory.

\begin{lstlisting}[language=C++, label={lst:lapsh}, caption={Laplacian evaluating using shared memory with boundaries condition}]
if (i >= 0 && i < NX && j >= 0 && j < NY){
    __shared__ double lapS[ THREADS_SHARED * THREADS_SHARED];
    lapS[sIdx] = deltam[Index];
    __syncthreads();

    if (threadIdx.x >= 2 && i < threadIdx.x - blockDim.x -2){ //shared
       lapy[idx] = - lapS[sIdx + 2] / 12.0 + 4.0 * lapS[sIdx + 1] / 3.0
			  	   - 5.0 * lapS[sIdx] / 2.0
			  	   - lapS[sIdx - 2] / 12.0 + 4.0 * lapS[sIdx - 1] / 3.0;
	else{ //global memory
		lapy[idx] = - deltam[idx + 2] / 12.0 + 4.0 * deltam[idx + 1] / 3.0
			  		- 5.0 * deltam[idx] / 2.0
			  		- deltam[idx - 2] / 12.0 + 4.0 * deltam[idx - 1] / 3.0;
	}
}
\end{lstlisting}

The current approach seems very promising for reducing global memory. However, great amount of time is spent on loading data onto the shared memory array, In consequence delaying threads executing, resulting a decrease in performance. Fast allocating shared memory data is the optimal solution to ensure the optimal use of this type of memory. The results of such implementation are in the following section, optimization results.

\subsection{Structure of Arrays, SAO}

AoS and SoA refer to "Array of Structures" and "Structure of Arrays" respectively. These two terms refer to two different ways of laying out your data in memory. This is illustrated in figure \ref{fig:aos} and \ref{fig:sao} respectively. AOS, grouping properties of an object together and making an array of those objects in memory, whereas a structure of arrays would be a single structure in which you make an array for each property. The structure of arrays can allow for better cache utilization, easier to access continues data, making better use of each read you make from memory, providing a more effective route to memory. 

\begin{figure}[htbp]
	\centering
		\includegraphics[width=0.68\textwidth]{Figures/aos.png}
		\smallskip
	\caption[Array of structures (AOS)]{AOS memory layout }
	\label{fig:aos}
\end{figure}


\begin{figure}[htbp]
	\centering
		\includegraphics[width=0.68\textwidth]{Figures/soa.png}
		\rule{35em}{0.2pt}
	\caption[Structure of Arrays (SAO)]{SAO memory layout}
	\label{fig:sao}
\end{figure}

The initial implementation the  x, y, z data was allocated in separated blocks. Furthermore when accessing blocks of the the same coordinates, the register access the data as the figure\ref{fig:aos}.

\begin{lstlisting}[language=C++, caption={AOS implementation}]
deltam_x = (double **)calloc(NYCUDA, sizeof(double *));
deltam_y = (double **)calloc(NXCUDA, sizeof(double *));
deltam_z = (double **)calloc(NXCUDA, sizeof(double *));
\end{lstlisting}

To solve the issue, a custom class GPUMatrix was programmed. The class contained all the matrices for the device. Moreover, the classes allocated the data for each Matrix and free the memory automatically when the simulation is over. The was allocated in a structure that easier for the device to access common elements. For example, evaluating operations only on the x coordinate, the kernel physically access matrices that are near by. Eliminating unassary shift in registers, or hierarchy memory access.

\begin{lstlisting}[language=C++, caption={SOA implementation}]
    GPUMatrix<T> *dev_deltam;
    GPUMatrix<T> *dev_sdex; //Exchange term
    GPUMatrix<T> *dev_sfrelax;
    GPUMatrix<T> *dev_m; 
\end{lstlisting}

The current approach eliminates unnecessary data shift in registers, and is able to stack more values per registers. In theory more computational time per threads. The results of such implementation is illustrate on the last section of the chapter. With the new implementation the code become more readable for future improvements. 

\subsection{Occupancy}
 
Firstly, we increased the use of constant memory in the device, eliminating redundant evaluations of variables and operations. The results are increase in performance and more computational workload on each thread. In addition, constant memory modifications are illustrated in the code \ref{lst:constant}. Matrix calculation for the boundary conditions \ref{eq:matrix3} and \ref{eq:matrix3} were implemented using constant Memory, reducing unnecessary calls in the device.
 
 \begin{lstlisting}[language=C++,  label={lst:constant}, caption={Constant Memory changes}]
 gsource << <blocks, threads >> >(u_val, dev_sm_z, dev_mz, NXCUDA);
 
 sfrelax_y[index] = -deltam_y[index] / tau_sf;
     
 DELTAX = (double)TX / (double)NX;
\end{lstlisting}
 
The different numbers of threads per block and as well the number of blocks per grid can dramatically increase or decrease the performance of the application. The table \ref{tab:ocu} illustrate the different threads per block configuration on the GeForce GTX 580. Using NVIDIA's Profiler is possible to obtain the occupancy percentage of threads the device. The initial configuration for the Fermi and the Kepler was 32x32 threads per block for global memory and 16x16 threads per block for the shared memory. We found, that the optimal configuration for the Fermi cards was 16x16 threads per block and as well for the shared memory and for the Kepler cards was 32x32 threads per block for both memory types. 

\begin{table}[h]
\centering
  \begin{tabular} { | l | l | l | l | l | l | }
    \hline
    Threads & Shared & speedup & time & Occupancy \\
    \hline
     8x8 &  8x8 & 7.217x & 52318.3  & 56.6\% \\
    \hline
     16x16 & 8x8 & 7.625x & 49517.3 & 86.6\% \\
    \hline
    16x16 & 16x16 & 7.978x & 47329.2 & 100.0\% \\
    \hline
    32x32 & 16x16 & 7.356x & 51333.4 & 66.6\% \\
    \hline
    32x32 & 32x32 & \multicolumn{3}{|c|}{Failed}\\
    \hline
  \end{tabular}
  \caption{Threads per block configuration and occupancy on the Fermi architecture}
  \label{tab:ocu}
  \end{table}


\section{Optimization results}

This section is a overview of the optimization results compared with the initial CUDA implementation. Each version of the code is compared with the first test results. The figure \ref{fig:gpuop} and \ref{fig:speedup} illustrates the the time execution and the speedup respectively for the tables \ref{tab:time} and \ref{tab:speed}. The final version of the code is the Occupancy. Moreover, the greatest performance occurred on the GeForce GTX 580 Card with 8.0x speedup \ref{fig:speedup}.

\begin{table}[h]
\centering
  \begin{tabular} { |  l  |  l | l  |  l  | l | l | l |}
    \hline
    GPU & Original & Constant & Streams & Shared & SAO & Occupancy \\
    \hline
    Tesla K20m & 107322.7 & 101513.4 & 97106.0 & 90201.7 & 68988.2 & 66456.0\\
   \hline
    Tesla M2070 & 110912.3 & 103212.4 & 130754.1 & 97343.4 & 73938.1 & 70299.3\\
    \hline
    Tesla C2050 & 109635.1 & 101212.4 & 128516.6 & 96762.0 & 72964.5 & 69358.1\\
   \hline
    GeForce GTX 580 & 70002.7 & 68712.2 & 76481.9 & 68567.1 & 51603.7 & 47213.2\\
   \hline
    GeForce 650m & 244372.9 & 237371.9 & 227237.8 & 279804 & 181217.4 & 174419\\
   \hline
  \end{tabular}
    \caption{GPU Optimization time}
  \label{tab:time}
  \end{table}
  
  The table \ref{tab:time} displays the overall executing time for all the version of the code, original, constant, streams, shared memory, SAO and Occupancy. We can see compared the initial time and the final there is a difference between 40s and 50s time decrease. The time reduction is relatively low. There is no big difference in waiting 100s or 66s to a simulation to complete. However, if we increase the data set to five decimal points, the simulation can take up-to a couple of hours or days. Finally, the speedup comparison in table \ref{tab:speed} and figure \ref{fig:speedup} illustrates how much performance increase we could obtain in a new simulation using the new implementation.
  
  \begin{table}[h]
\centering
  \begin{tabular} { |  l  |  l | l  |  l  | l | l | l |}
    \hline
    GPU & Original & Constant & Streams & Shared & SAO & Occupancy\\
    \hline
    Tesla K20m & 3.517x & 3.718x & 3.888x & 4.186x & 5.473x & 5.682x\\
   \hline
    Tesla M2070 & 3.403x & 3.534x & 2.888x & 3.879x & 5.107x & 5.371x\\
    \hline
    Tesla C2050 & 3.442x & 3.571x & 2.938x & 3.902x & 5.175x & 5.444x\\
   \hline
    GeForce GTX 580 & 5.391x & 5.521x & 4.937x & 5.551x & 7.317x & 8.0x\\
   \hline
    GeForce 670MX & 1.544x & 1.598x & 1.662x & 1.349x & 2.084x & 2.163x\\
   \hline
  \end{tabular}
    \caption{GPU Speedup performance}
  \label{tab:speed}
  \end{table}

\begin{figure}[htbp]
	\centering
		\includegraphics[width=0.98\textwidth]{Figures/gpuOptimization.png}
		\smallskip
	\caption[Overall simulation time]{Overall simulation time}
	\label{fig:gpuop}
\end{figure}

The table \ref{fig:final} illustrates the final profiling results using NVIDIA's profiling tools. On the initial profiling \ref{fig:initial}, the Laplacian evaluation consummend about half of the overall simulation time. However, on the final optimization results\ref{fig:final}, the Laplacian was reduced from 44.37$\%$ to 26.24 $\%$ on execution time. But more importantly the importance of the kernel was reduced, in other words more computational workload on the kernels. The same occurred for the Runge and Kutta term evaluation. The speedup mainly occurred in the Occupancy version of the code, see table \ref{tab:speed}. Furthermore, the {\listf gsdExchangeFull} incremented from 13.37$\%$. 23.35$\%$, which is not necessary good. The increment in time is due to shift in streams operators, the {\listf gsdExchangeFull} is processed in the default stream zero, while the others kernels launched concurrently in a different stream.
\begin{figure}[htbp]
	\centering
		\includegraphics[width=1.0 \textwidth]{Figures/speedup.png}
		\smallskip
	\caption[Speedup performance output]{Speedup performance output.}
	\label{fig:speedup}
\end{figure}

\begin{figure}[htbp]
	\centering
	  \begin{tabular} { |  l  |  l | l  | l | l |}
	      \hline
	    Time $\%$& Time & Calls & Avg & Kernel \\
    \hline
   35.24 & 21.33s & 216330 & 98.5us & {\listf gSolTermRK4Relaxation } \\
   \hline
   26.24 & 15.88s & 288441 & 55.0us &  {\listf glaplacianXYBroundaries }\\
   \hline
   23.35 & 14.13s & 160000 & 88.3us & {\listf gsdExchangeFull} \\
   \hline
   15.17 & 9.18s & 72108 & 127.2us & {\listf gSolTerm4RK4Relaxation }\\ 
   \hline
    \end{tabular}
	\caption[Optimization results with the Profiler]{Final optimization results using NVIDIA's profiler, on the Tesla K20m}
	\label{fig:final}
\end{figure}
    
The Tesla K20m was the only GPU which in every code modification it did not lose performance over the course of the optimization process. However, the other GPUs drop performance in the stream optimization stage. The stream process is where each kernel was divided into three separated kernels. Doing this we were able to calculate the x, y, z coordinates independently. In addition, this enable room to implement shared memory across possible kernels. The GPU table specifications \ref{tab:gpus} illustrates that the Tesla K20m is the only GPU card with CC of 3.5. Therefore, has access to Hyper-Q technology. which is able to synchronize automatically concurrent kernels, just by activated a steam process to the kernel.

The SOA optimization, improved overall dramatically the performance of the application, obtaining a 1.2x - 2.0x speedup in all GPU cards, see table \ref{tab:speed} and figure \ref{fig:speedup}. The final version, Occupancy, improved up-to 0.7x speedup on the 580 GPU. However, for the Teslas cards only 0.2x-0.25x speedup. The speedup difference, 0.5x,  is due to the process cycle of the GeForce GPU. 

\begin{figure}[htbp]
	\centering
		\includegraphics[width=1.0\textwidth]{Figures/speed.png}
		\smallskip
	\caption[Optimization speedup overview]{Optimization speedup overview}
	\label{fig:speeduplast}
\end{figure}


We expected that the newest card, the Tesla k20m, would obtain the highest speedup overall, certainly because it has more CUDA cores, the highest compute capabilities. However, if falls behind the GeForce 580 with about 2.5x speedup difference (table \ref{tab:speed}). In addition, the Tesla K20 only had a difference of 0.3x speedup compared with the Teslas Cards, see figure \ref{fig:speedup} and \ref{fig:speeduplast}. The GeForce compared with the others GPUs specifications has most Processor clock (GHz), more mathematical calculations per cycle, se table \ref{tab:gpus} for GPU specifications comparison. The results demonstrate the increase of workload and occupancy on the device. Increasing the computational process per thread and per block.


  \vspace{4.0em}

Finally, the techniques and practices from chapter \ref{Heterogeneous Performance Analysis and Practices} were used to archive speedup and increase performance on the initial CUDA implementation. Methods such as: increase the use of constant memory, shared memory, changed the memory allocating access, analyzed thread branching and finally analyzed kernel occupancy. The highest performance of all GPUS did not occurred on the newest NVIDIA card, the K20m, thus, the most expensive of all the GPUs available in the cluster. The actual improvement appeared on the GeForce 580 (the more GHz of all GPUs) with a 2.32x speedup difference in comparison with the Tesla K20m, table \ref{fig:speedup}. If the problem data set is to be scaled, for example, to a simulation of 8 days, the speedup performance of 8.0x will drastically decrease the time execution in only one day. Moreover, is possible to execute more simulations on the initial time step.
  

% Chapter Template

\chapter{Conclusions and future work} % Main chapter title

\label{Conclusions and future work} % Change X to a consecutive number; for referencing this chapter elsewhere, use \ref{ChapterX}

\lhead{Chapter 6. \emph{Conclusions and future work}} % Change X to a consecutive number; this is for the header on each page - perhaps a shortened title

%----------------------------------------------------------------------------------------
%	SECTION 1
%----------------------------------------------------------------------------------------


GPUs definitely have a place in the world of computational physics, their use allows to do the same work with less energy and more science with less resources. They make HPC computing clusters affordable for small research groups. The true limit test of this new technology will be if it is actually used to advance new science. In the field of computational physics studies that do push the barrier of what is computationally feasible, from speedups of 1.5x to 20x using GPUs\cite{applications}.

Acceptance has been slow due to many factors, GPUs are sometimes seen as a fad or a niche, the specialized skill set and effort required for GPU programming along with the risk of spending money to setup a GPU cluster, does raise a concern for productivity and viability of this technology. Adopting this technology requires abandoning legacy codes and smart optimizations that have been developed over the years. A wrong choice may result in wasted time and effort.

What is certain is at the moment, is the overall direction of the industry towards higher parallelism, as single threaded performance has reached a local limit, all types of processors are seeking more performance out of parallelism. This means that a large portion of the work needed to parallelize a code for a certain parallel architecture will most probably be applicable to another parallel architecture as well. From the literature and my experiences, one can observe that in order to achieve good results in programming with GPUs it is necessary to take a Heterogeneous approach to coding. That is adopting multi-threaded CPUs and concurrent GPU type algorithms.

Spintronics. In particular we are involved in designing new magnetic materials for spin-devices and modeling and understanding of spin-transport at molecular and atomic scale. By computer simulation is possible to predict their output. Furthermore prove the theoretical experiments.

In the simu


The current thread is to push the hardware capabilities and performance along with Mooers' Law, despite these issues there are some trends in the hardware industry that will make working with GPU easier and more widespread within a HPC context:

\begin{description}
  \item[3D Memory] \hfill \\
 Stacks DRAM chips into dense modules with wide interfaces, and brings them inside the same package as the GPU. This lets GPUs get data from memory more quickly – boosting throughput and efficiency – allowing us to build more compact GPUs that put more power into smaller devices. The result: several times greater bandwidth, more than twice the memory capacity and quadrupled energy efficiency.
  
  \item[NVLink] \hfill \\
 Today’s computers are constrained by the speed at which data can move between the CPU and GPU. NVLink puts a fatter pipe between the CPU and GPU, allowing data to flow at more than 80GB per second, compared to the 16GB per second available now.
 
 \item[Pascal Module] \hfill \\ 
  NVIDIA has designed a module to house Pascal GPUs with NVLink. At one-third the size of the standard boards used today, they’ll put the power of GPUs into more compact form factors than ever before.
  
   \item[Mobile and embedded Devices] \hfill \\ 
   Kepler
   Erista Maxwell GPU
  
   \item[Cloud Computing] \hfill \\ 

  \end{description}


To conclude, I offer my personal perspective on GPU computing. I think the importance of using accelerator hardware is an economic and environmental issue. The environmental aspect of doing computing is often overlooked, but an ever increasing important one. As heavy computer users we will have to take responsibility for our electricity use. The benefit of less energy use is clear.



%----------------------------------------------------------------------------------------
%	THESIS CONTENT - APPENDICES
%----------------------------------------------------------------------------------------

%\addtocontents{toc}{\vspace{2em}} % Add a gap in the Contents, for aesthetics

%\appendix % Cue to tell LaTeX that the following 'chapters' are Appendices

% Include the appendices of the thesis as separate files from the Appendices folder
% Uncomment the lines as you write the Appendices

%\input{Appendices/AppendixA}
%\input{Appendices/AppendixB}
%\input{Appendices/AppendixC}

%\addtocontents{toc}{\vspace{2em}} % Add a gap in the Contents, for aesthetics

%\backmatter

%----------------------------------------------------------------------------------------
%	BIBLIOGRAPHY
%----------------------------------------------------------------------------------------



\nocite{*}

\lhead{\emph{Bibliography}} % Change the page header to say "Bibliography"

\bibliographystyle{abbrv} % Use the "unsrtnat" BibTeX style for formatting the Bibliography

\bibliography{Bibliography} % The references (bibliography) information are stored in the file named "Bibliography.bib"

\label{Bibliography}

\end{document}  